%% Generated by Sphinx.
\def\sphinxdocclass{report}
\documentclass[a4paper,14pt,italian]{sphinxmanual}
\ifdefined\pdfpxdimen
   \let\sphinxpxdimen\pdfpxdimen\else\newdimen\sphinxpxdimen
\fi \sphinxpxdimen=.75bp\relax

\PassOptionsToPackage{warn}{textcomp}
\usepackage[utf8]{inputenc}
\ifdefined\DeclareUnicodeCharacter
% support both utf8 and utf8x syntaxes
  \ifdefined\DeclareUnicodeCharacterAsOptional
    \def\sphinxDUC#1{\DeclareUnicodeCharacter{"#1}}
  \else
    \let\sphinxDUC\DeclareUnicodeCharacter
  \fi
  \sphinxDUC{00A0}{\nobreakspace}
  \sphinxDUC{2500}{\sphinxunichar{2500}}
  \sphinxDUC{2502}{\sphinxunichar{2502}}
  \sphinxDUC{2514}{\sphinxunichar{2514}}
  \sphinxDUC{251C}{\sphinxunichar{251C}}
  \sphinxDUC{2572}{\textbackslash}
\fi
\usepackage{cmap}
\usepackage[T1]{fontenc}
\usepackage{amsmath,amssymb,amstext}
\usepackage{babel}



\usepackage{times}
\expandafter\ifx\csname T@LGR\endcsname\relax
\else
% LGR was declared as font encoding
  \substitutefont{LGR}{\rmdefault}{cmr}
  \substitutefont{LGR}{\sfdefault}{cmss}
  \substitutefont{LGR}{\ttdefault}{cmtt}
\fi
\expandafter\ifx\csname T@X2\endcsname\relax
  \expandafter\ifx\csname T@T2A\endcsname\relax
  \else
  % T2A was declared as font encoding
    \substitutefont{T2A}{\rmdefault}{cmr}
    \substitutefont{T2A}{\sfdefault}{cmss}
    \substitutefont{T2A}{\ttdefault}{cmtt}
  \fi
\else
% X2 was declared as font encoding
  \substitutefont{X2}{\rmdefault}{cmr}
  \substitutefont{X2}{\sfdefault}{cmss}
  \substitutefont{X2}{\ttdefault}{cmtt}
\fi


\usepackage[Sonny]{fncychap}
\ChNameVar{\Large\normalfont\sffamily}
\ChTitleVar{\Large\normalfont\sffamily}
\usepackage{sphinx}

\fvset{fontsize=\small}
\usepackage{geometry}

% Include hyperref last.
\usepackage{hyperref}
% Fix anchor placement for figures with captions.
\usepackage{hypcap}% it must be loaded after hyperref.
% Set up styles of URL: it should be placed after hyperref.
\urlstyle{same}
\addto\captionsitalian{\renewcommand{\contentsname}{Indice:}}

\usepackage{sphinxmessages}
\setcounter{tocdepth}{1}



\title{Mastrangelo Sindaco}
\date{24 apr 2019}
\release{}
\author{2019, Elezioni Amministrative - Gioia del Colle}
\newcommand{\sphinxlogo}{\sphinxincludegraphics{logo.png}\par}
\renewcommand{\releasename}{}
\makeindex
\begin{document}

\ifdefined\shorthandoff
  \ifnum\catcode`\=\string=\active\shorthandoff{=}\fi
  \ifnum\catcode`\"=\active\shorthandoff{"}\fi
\fi

\pagestyle{empty}
\sphinxmaketitle
\pagestyle{plain}
\sphinxtableofcontents
\pagestyle{normal}
\phantomsection\label{\detokenize{index::doc}}


Il programma politico per Giovanni Mastrangelo Sindaco di Gioia del Colle è nato dall’ascolto dei nostri cittadini, confrontandosi con chi vive e lavora nella nostra città.
Esprime cosa la città vuole, nato dalla partecipazione e dialogo dove ogni cittadino è protagonista della sua città.

\noindent{\hspace*{\fill}\sphinxincludegraphics[width=1.000\linewidth]{{banner}.png}\hspace*{\fill}}


\chapter{Efficienza e riorganizzazione della macchina amministrativa}
\label{\detokenize{comune_efficiente:efficienza-e-riorganizzazione-della-macchina-amministrativa}}\label{\detokenize{comune_efficiente::doc}}
\noindent{\hspace*{\fill}\sphinxincludegraphics[width=1.000\linewidth]{{comune}.jpg}\hspace*{\fill}}

Servizi pubblici più efficienti significa migliorare la vita delle persone e i suoi fabbisogni, è la prima e la più importante missione da non perdere mai di vista.

Un Comune come Gioia del Colle non può continuare ad operare in carenza di organico così come avvenuto fino ad oggi.
La prima sfida da affrontare sarà quella di \sphinxstylestrong{riorganizzare le aree e i servizi attraverso una puntuale e rigorosa individuazione dei compiti e dei centri di responsabilità}.

\sphinxstylestrong{Utilizzare tutte le risorse economiche rinvenienti dal personale prossimo al pensionamento} per procedere ad un adeguato ricambio generazionale che porti nuove competenze per modernizzare i servizi ai fabbisogni dei cittadini.

Il nuovo personale dovrà essere individuato attraverso profili professionali altamente qualificati in grado di portare quella innovazione tecnologica per consentire all’ente un salto di qualità.
E’ necessario ridurre l’apparato burocratico con una cultura dell’organizzazione dei servizi comunali che permetta di raggiungere un modello di tipo manageriale, basato su competenza, responsabilità e orientamento al risultato per perfezionare la risposta, il benessere e la qualità della vita dei cittadini e aziende.

L’impegno sarà rivolto a raggiungere un elevato livello di gestione digitale dei servizi per ridurre al minimo discrezionalità e tempi di attesa.

L’esperienza dei dipendenti prossimi al pensionamento dovrà essere trasmessa attraverso una \sphinxstyleemphasis{fase formativa} per i nuovi assunti affinchè non si perda il bagaglio di esperienza e conoscenza maturata negli anni.


\section{Limitazioni assunzioni part-time}
\label{\detokenize{comune_efficiente:limitazioni-assunzioni-part-time}}
Per garantire un impiego pieno ed efficace dei dipendenti bisognerà ridurre e limitare l’impiego di personale part-time.


\section{Obiettivo di Gestione}
\label{\detokenize{comune_efficiente:obiettivo-di-gestione}}
Le risorse destinate alle premialità in favore dei funzionari saranno erogate al pedissequo raggiungimento degli obiettivi raggiunti previa attenta ed oggettiva valutazione.


\section{Prenotazioni online}
\label{\detokenize{comune_efficiente:prenotazioni-online}}
Sarà istituito un servizio di prenotazione online per i cittadini, imprese e tecnici affinchè Responsabili delle aree e dei servizi siano disponibili al ricevimento.
Lo stesso servizio permetterà di richiedere documenti che potranno essere prenotati online e ritirati in data certa stabilita dall’ufficio competente.


\section{Ufficio reclami}
\label{\detokenize{comune_efficiente:ufficio-reclami}}
Al fine di migliorare la qualità ed efficienza dei servizi resi agli utenti sarà istituito un Uffico Reclami.
Attraverso il quale i fuitori dei servizi, previa identificazione, potranno segnalare problemi inefficienze e ritardi da parte degli uffici.
Tale servizio consentirà una idonea ed oggettiva valutazione delle criticità avvertite dall’utenza.


\section{Ufficio Progettazione e bandi}
\label{\detokenize{comune_efficiente:ufficio-progettazione-e-bandi}}
Al fine di captare tutte quelle risorse economiche rinvenienti da bandi europei, regionali e sovracomunali sarà istituito un ufficio con questa specifica funzione.
Tale ufficio avrà il compito ulteriore di seguire tutte le fasi di partecipazione ai bandi dalla progettazione, monitoraggio delle scadenze sino alla corretta rendicontazione economica.


\section{Potenziamento SUAP}
\label{\detokenize{comune_efficiente:potenziamento-suap}}
Potenziamento degli Sportelli Unici per le Attività Produttive (SUAP) al fine di consentire una riduzione dei tempi burocratici garantendo una rapida risposta alle esigenze di imprese e attività commerciali.


\chapter{Trasparenza}
\label{\detokenize{trasparenza:trasparenza}}\label{\detokenize{trasparenza::doc}}
\noindent{\hspace*{\fill}\sphinxincludegraphics[width=1.000\linewidth]{{trasparenza2}.jpg}\hspace*{\fill}}

\sphinxstylestrong{La trasparenza senza nessuna restrizione giuridica non sarà più un mero adempimento burocratico, ma diventerà la regola generale e valore fondamentale.}

L’obiettivo della coalizione sarà quello di rendere quanto più efficace la conoscenza e la lettura semplice dei dati e documenti pubblici da parte dei cittadini offrendo siti istituzionali leggibili e percorsi semplificati per l’accesso alle informazioni (\sphinxhref{http://www.foiapop.it/ente/6f46f254-0ff0-40cd-be4f-f354ee3ccc88/scegli}{FOIAPop}).
La rendicontazione periodica sarà un nostro impegno per una comunicazione pubblica in grado di favorire sia una trasparenza dell’agire amministrativo, sia una partecipazione dei cittadini alla vita pubblica.
Sostenere anche economicamente le attività inerenti servizi di e-government, formazione ed educazione alla trasparenza, al riuso e richieste di dati.


\section{Codice Etico}
\label{\detokenize{trasparenza:codice-etico}}
Il Comune adotterà il \sphinxhref{http://www.avvisopubblico.it/home/wp-content/uploads/2014/05/20140715\_comune\_san-miniato\_giunta\_cartadipisa.pdf}{Codice Etico degli Amministratori}, con l’intento di assicurare e testimoniare la trasparenza, l’integrità e la legalità nelle attività dell’Ente, contrastando ogni possibile forma di corruzione e di infiltrazione criminosa.
Aggiornare lo Statuto Comunale affinchè ci sia un impegno concreto degli Amministratori ai valori della Costituzione della Repubblica Italiana ed ai principi di fedeltà allo Stato, attuando concretamente il principio della separazione tra i poteri di indirizzo e di controllo politico-amministrative.


\chapter{Tributi}
\label{\detokenize{tributi:tributi}}\label{\detokenize{tributi::doc}}
\noindent{\hspace*{\fill}\sphinxincludegraphics[width=1.000\linewidth]{{economia}.jpg}\hspace*{\fill}}

Negli ultimi anni il servizio di riscoissione tributi nonostante fosse gestito con il supporeto di una società privata esterna, non ha garantito un’efficace attività di riscossione che ha determinato l’accumularsi di crediti insoluti.
Tale inefficienza ha impedito all’ente dfi utilizzare le risorse in entrata per svolgere le proprie funzioni basilari.
Obiettivo primario sarà quello di monitorare l’attività di recupero efficacemente in tutte le fasi indispensabioli all’effettivo incasso delle somme dovute.

Constrastare il fenomeno dell’evasione dei tributi comunali per un sistema fiscale più equo e sostenibile per famiglie e imprese.

Al contempo bisognerà agevolare le imprese che presidiano e animano la città valorizzando e sviluppando l’economia locale, garantendo il lavoro e la possibilità per le famiglie di vivere e progredire nel benessere.
Sui cittadini dovrà ricadere il beneficio derivante dalla elevata percentuale di raccolta differenziata atteso che gli stessi hanno consentito il raggiungimento di importanti obiettivi.
Garantire alle attività commerciali e imprenditoriali una immediata semplificazione nel pagamento delle tasse derivanti dall’occupazione di suolo pubblico ponendo come obiettivo il raggiungimento anche in tal senso di una procedura telematica per la quantificazione e relativo pagamento.

L’Amministrazione avrà l’obiettivo di monitorare l’attuazione dell’indirizzo politico per il recupero dell’evasione tributi, il rispetto del contratto con la società di riscossione, di elaborare e pubblicare report periodici di data analysis sullo stato di evasione/recupero fiscale nel rispetto della tutela dei dati personali.

Impegno prioritario sarà altresì la sottoscrizione del protocollo d’intesa con gli \sphinxstylestrong{Uffici della Agenzia delle Entrate}, unitamente alla società incaricata alla riscossione tributi.

La comunicazione costante e continua tra l’Ufficio Tributi, Ufficio Tecnico e Servizio Ragioneria sarà la chiave strategica per garantire una rigorosa e attendibile fotografia dello stato di salute economico dell’Ente.

Obiettivo a medio/lungo termine sarà quello di potenziare l’Ufficio Tributi al fine di raggiungere una piena autonomia nella gestione della riscossione dei tributi.


\section{Riferimenti}
\label{\detokenize{tributi:riferimenti}}\begin{enumerate}
\def\theenumi{\arabic{enumi}}
\def\labelenumi{\theenumi .}
\makeatletter\def\p@enumii{\p@enumi \theenumi .}\makeatother
\item {} 
Ai sensi dell’art. 2 del proprio \sphinxhref{https://www.agenziaentrateriscossione.gov.it/export/it/Gruppo/statuto-AdE-Risc.pdf}{Statuto}, l’Agenzia delle Entrate, nel perseguimento dei propri fini istituzionali, assicura la collaborazione con il sistema delle autonomie locali e promuove e fornisce servizi agli enti locali per la gestione dei tributi di loro competenza, stipulando convenzioni per la liquidazione, l’accertamento e la riscossione di tali tributi;

\item {} 
L’art. 1 del Decreto-legge 30 settembre 2005, convertito dalla Legge 2 dicembre 2005 n. 248, successivamente modificato dal \sphinxhref{http://www.gazzettaufficiale.it/gunewsletter/dettaglio.jsp?service=1\&datagu=2010-05-31\&task=dettaglio\&numgu=125\&redaz=010G0101\&tmstp=1275551085053}{Decreto Legge 31 Maggio 2010 n.78}, prevede, nell’ambito dell’attivita” di contrasto all’evasione, la partecipazione dei Comuni all’accertamento;

\item {} 
Il Provvedimento del Direttore dell’Agenzia delle Entrate n. \sphinxhref{https://toscana.agenziaentrate.it/sites/toscana/files/public/2008/rapporti\%20con\%20enti/A4\%20-\%20Provvedimento\%20Direttoriale\%203\%20dicembre\%202007.pdf}{187461 del 3 dicembre 2007} ha introdotto una prima disciplina riguardante le modalita” con le quali dovra” realizzarsi la partecipazione dei Comuni all’attivita” di accertamento;

\item {} 
Il provvedimento del Direttore dell’Agenzia n. \sphinxhref{https://umbria.agenziaentrate.it/sites/umbria/files/public/NormativaCollaborazioneComuni/Provvedimento+26+novembre+2008.pdf}{175466 del 26 novembre 2008} ha definito le modalita” tecniche relative alla trasmissione da parte dei Comuni delle informazioni suscettibili di utilizzo ai fini dell’accertamento;

\item {} 
Il \sphinxhref{http://www.gazzettaufficiale.it/eli/id/2008/08/21/08A05897/sg}{Decreto Legge n.112 del 25 giugno 2008} ha dettato disposizioni urgenti per la stabilizzazione della finanza pubblica e la perequazione Tributaria;

\end{enumerate}


\chapter{Urbanistica}
\label{\detokenize{urbanistica:urbanistica}}\label{\detokenize{urbanistica::doc}}
\noindent{\hspace*{\fill}\sphinxincludegraphics[width=1.000\linewidth]{{urbanistica}.jpg}\hspace*{\fill}}

L’attività urbanistica deve essere improntata alla riduzione del consumo del suolo e contemporaneamenete correlata all’implementazione dei servizi da offrire ai cittadini.
Gli interventi in materia urbanistica devono sempre garantire il miglioramento della qualità urbana.

La pubblica amministrazione e gli attori economici dovranno uniformarsi al principio della perequazione nella pianificazione urbanistica contemperando le rispettive esigenze al fine di raggiungere un modello di sviluppo che abbia come obiettivo il recupero e riuso delle aree dismesse, la riqualificazione di aree degradate nel rispetto dell’ambiente urbano.

Obiettivo primario deve indirizzarsi al completamento del PUG (Piano Urbanistico Generale) al fine di garantire un’approcio moderno per lo sviluppo del territorio nei prossimi anni.
Nelle more dell’approvazione sarà istituito un ufficio in grado di promuovere una collaborazione tra pubblico e privato nell’ambito di progetti di rigenerazione urbana.

Compito dell’Amministrazione sarà quello di attivare procedure che le norme nazionali e regionali consentono di utilizzare per la risoluzione di problemi urbanistici e per lo sviluppo delle aree produttive della città.
Promuovere accordi di programma al fine di utilizzare strumenti perequativi per la realizzazione di nuovi standard urbanistici per affrontare i problemi relativi alle aree commerciali e artigianali.

Approvazione immediata del \sphinxstylestrong{RET (Regolamento Edilizio Tipo)} della Regione Puglia al fine di garantire una corretta e puntuale applicazione della normativa edilizia, eliminando dubbi e interpretazioni discrezionali.


\section{Centro storico}
\label{\detokenize{urbanistica:centro-storico}}
Il Centro Storico deve rappresentare l’anima e l’identità urbanistica della nostra città.
Per ridare bellezza e autenticità, nel rispetto delle origini urbanistiche del centro storico, sarà obiettivo primario l’approvazione del \sphinxstylestrong{Regolamento Comunale Piano Colori e Materiali}.
Tale regolamento avrà come obiettivo quello di uniformare ed armonizzare tutti gli interventi di riqualificazione aventi ad oggetto edifici ricadenti nella zona storica.
Utilizzo ed impiego dei manufatti presenti nel centro storico rimaranno a destinazione libera evitando qualsivoglia inutile adempimento burocratico al fine di incentivare interventi ed insediamenti commerciali.


\section{Ufficio Tecnico efficiente}
\label{\detokenize{urbanistica:ufficio-tecnico-efficiente}}
Negli ultimi anni gli uffici comunali hanno subito troppi avvicendamenti di personale responsabile che hanno compromesso l’efficienza degli stessi.
L’amministrazione dovrà avvalersi delle migliori competenze sukl mercato a cui affidare la responsabilità dell’Ufficio Tecnico.


\section{Viabilità}
\label{\detokenize{urbanistica:viabilita}}

\subsection{«Piano Marshall» per la manutenzione delle strade}
\label{\detokenize{urbanistica:piano-marshall-per-la-manutenzione-delle-strade}}
\sphinxstylestrong{Piano Marshall per la manutenzione stradale}, cronoprogramma delle priorità di interventi mirati volti a garantire sicurezza stradale e decoro urbano.
Una buona parte delle somme ricavate dal recupero dei tributi locali evasi deve essere impiegata per la manutenzione stradale, diventata ormai un emergenza inprorogabile.

Controllo di interventi di sui tratti urbani da parte di società esterne (Enel, AQP, Società telefoniche, ecc.) che spesso concorrono al degrado della viabilità stradale.
Per tanto sarà necessario acquisire una preventiva comunicazione degli interventi che le società eseguiranno sul territorio, al fine di pianificare con puntualità e senza dispendio di risorse gli interventi di manutenzione.
Al termine dei lavori suddetti saranno eseguiti controlli rigorosi sugli interventi e qualità dei materiali impiegati nel rifacimento della strade.

\noindent{\hspace*{\fill}\sphinxincludegraphics[width=1.000\linewidth]{{buca_1}.jpg}\hspace*{\fill}}

\noindent{\hspace*{\fill}\sphinxincludegraphics[width=1.000\linewidth]{{buca_2}.jpg}\hspace*{\fill}}


\subsection{Sicurezza}
\label{\detokenize{urbanistica:sicurezza}}
La messa in sicurezza di tratti urbani sensibili e ad alta frequentazione (Centri sportivi, scuole, ecc.) saranno oggetto di intervento attraverso l’utilizzo di dissuasori, rallentatori e nuove rotatorie.

\noindent{\hspace*{\fill}\sphinxincludegraphics[width=1.000\linewidth]{{rallentatore_2}.jpg}\hspace*{\fill}}

\noindent{\hspace*{\fill}\sphinxincludegraphics[width=1.000\linewidth]{{viale_einaudi}.jpg}\hspace*{\fill}}


\section{Cittadinanza attiva e gestione degli spazi comuni}
\label{\detokenize{urbanistica:cittadinanza-attiva-e-gestione-degli-spazi-comuni}}
L’amministrazione dovrà porsi come obiettivo da raggiungere quello di sensibilizzare la cittadinanza alla cura degli spazi pubblici.
Bisognerà promuovere e sollecitare i privati (aziende, scuole, residenti, cittadini) ad adottare un luogo di uso comune (rotatorie, giardini, aree verdi, ecc.); sarà così possibile efficentare la manutenzione, abbellire urbanisticamente la città e al contempo diffondere la cultura del bene comune e/o semplicemente consentire uno spazio pubblicitario alle imprese.

Sarà promossa una campagna sui social network e attraverso le pagine istituzionali al fine di promuovere la collaborazione tra privati e pubblico.
Permetterà ai soggetti privati di compiere erogazioni libere nei confronti del Comune, superando i limiti del \sphinxhref{http://www.comune.gioiadelcolle.ba.it/cms/files/c0b3f884-3f80-4495-9c06-691639a435c2}{Regolamento sui Beni Comuni} con una donazione, sponsorizzazione, o altre forme di collaborazione contro il degrado urbano.


\chapter{Commercio e Attività produttive}
\label{\detokenize{commercio:commercio-e-attivita-produttive}}\label{\detokenize{commercio::doc}}
\noindent{\hspace*{\fill}\sphinxincludegraphics[width=1.000\linewidth]{{commercio}.jpg}\hspace*{\fill}}

\sphinxstylestrong{Il Comune deve incentivare la capacità di consumo dei cittadini nel proprio territorio per limitare la crisi economica delle attività commerciali locali insieme dalle politiche economiche aggressive delle grandi distribuzioni.}

Compito dell’Amministrazione deve essere quello di incentivare il commercio per aumentare le opportunità di sviluppo del settore dovranno passare attraverso un piano di defiscalizzazione per incentivare nuove aperture commerciali e individuazione di aree urbane strategiche da riqualificare per potenziare il commercio e i servizi al cittadino.
Sarà compito dell’Amministrazione favorire lo sviluppo di una rete intercomunale turistica con lo scopo di attrarre potenziali consumatori incentivando il passaggio nella nostra città.

Sarà importante operare in sinergia con l’ASL un’attività di armonizzazione della normativa urbanistica comunale e quella sanitaria, al fine di agevolare l’insediamento di nuove attività nel centro storico anche operando una iniziale e parzioale riduzione delle nuove attività commerciali.
Ulteriore problema da affrontare sarà l’ubicazione dei mastelli per la raccolta dei rifiuti delle attività commerciali, con il coinvolgimento delle associazioni di categoria, ASL e gestore del servizio di rifiuti.
Si può ipotizzare l’utilizzo di contenitori comuni, armonizzati nel contesto urbanistico per quelle attività che non hanno spazio dove allocare il suddetto contenitore.

Sperimentare una giornata al mese per il mercato settimanale in altre zone della città, in particolar modo nelle zone periferiche.


\section{Riduzione fiscale}
\label{\detokenize{commercio:riduzione-fiscale}}
A Gioia del Colle piccole attività commerciali subiscono un’elevatissima pressione fiscale a volte insostenibile e ingiustificata.
Obiettivo dell’Amministrazione sarà quello di ridurre progressivamente i tributi locali, attraverso i benefici della raccolta differenziata e la lotta all’evasione fiscale.
L’occupazione di suolo pubblico è una possibilità di guadagno che non deve essere vessattoria ma ridotta al minimo consentito.
Aumentando i consumi delle attività commerciali si auspica un beneficio economico generale.


\subsection{Centro Storico e Zone Commerciali}
\label{\detokenize{commercio:centro-storico-e-zone-commerciali}}
La valorizzazione del centro storico, sulla scorta delle più avvedute esperienze comunali limitrofe, significa soprattutto riqualificazione e incentivazione del sistema economico-produttivo-commerciale, significa anche una riqualificazione naturale degli immobili adibiti ad uso commerciale attraverso il ripristino delle facciate, l’adeguamento delle vetrine e dei varchi di accesso all’immobile.


\subsection{Attività artigianali}
\label{\detokenize{commercio:attivita-artigianali}}
Riduzione della tariffa per i locali destinati ad attività espositive e le aree scoperte utilizzate per attività artigianali.


\subsection{Contro la desertificazione urbana}
\label{\detokenize{commercio:contro-la-desertificazione-urbana}}
Gli affitti elevati, le svendite incessanti, temporary store e outlet che minano la leale concorrenza, tasse vessatorie, sempre meno parcheggi nei centri storici, incidono sulla chiusura di attività commerciali causando la desertificazione in intere zone urbane, compromettendo anche la sicurezza dei cittadini.
\sphinxstylestrong{Riduzione dell’aliquota IMU} per i proprietari di immobili non locati da \sphinxstyleemphasis{almeno 3 anni} che daranno in affitto gli spazi per l’apertura di nuove attività.


\section{Zona artigianale}
\label{\detokenize{commercio:zona-artigianale}}
Il Piano Insediamento Produttivo (PIP) della Zona Artigianale di Gioia del Colle è scaduto da anni, bloccando di fatto il completamento degli insediamenti nell’area.
Sarà indispensabile apportare nuove modifiche per garantirne l’adeguamento alle effettive esigenze imprenditoriali ed ampliando una zona che preveda un Area Servizi che ancora oggi manca.
Il nuovo PIP potrà prevedere la possibilità di estendere anche all’insediamento di nuove attività commerciali, attraverso una previsione all’interno del PUG, per rivitalizzare e rendere più «appetibile» la zona.

Migliorare la viabilità partendo da una mappatura digitale affinchè le aziende siano facilmente raggiungibili.

Potenziare e riattivare la \sphinxstylestrong{Fibra ottica} presente nella zona di proprietà comunale, affinchè le aziende popssano usufruire della alta velocità di connessione anche in prospettiva Industria 4.0.

Riattivare e potenziare il sistema di videosorveglianza per garantire la sicurezza delle aziende nella zona industriale.

Promuovere attività di formazione con le aziende locali a favore di giovani da preparare al mondo del lavoro.


\chapter{Cultura}
\label{\detokenize{cultura:cultura}}\label{\detokenize{cultura::doc}}

\section{Importanza e valorizzazione}
\label{\detokenize{cultura:importanza-e-valorizzazione}}
La vera Conoscenza è quanto mai basilare all’interno delle moderne società e nei vari territori d’appartenenza; la storia e le tradizioni delle comunità in essa si riconoscono ed ad essa guardano al fine di migliorare le sensibilità di ognuno, il progresso e gli stili di vita.

La città di Gioia del Colle è antica e pregna di così molteplici tradizioni, usanze e costumi da farne potenzialmente un importante volano di crescita ed innovazione sociale.

Personalità illustri del passato, insigni uomini e donne di cultura contemporanei, vive ed interessanti intelligenze, operose Associazioni Culturali e di servizio hanno da sempre apportato alla nostra Città proposte ed iniziative sempre interessanti e degne di nota.
Gioia del Colle ha bisogno di un programma culturale e di promozione turistica nuovo ed innovativo; il lavoro che è stato portato avanti, è la sommatoria di varie anime che hanno collaborato insieme alla stesura con l’apporto di idee e proposte, ma anche redatto dopo un attento ascolto delle varie Associazioni Culturali e di numerosi singoli uomini e donne di cultura ed operatori culturali del territorio. Lo stesso, inoltre, è stato stilato nel solco della tradizione di Gioia del Colle e nel rispetto delle consuetudini, gusti, costumi, desideri, grado d’istruzione della popolazione, numero di Istituti Scolastici, consapevoli tutti che il vero Sapere è alla base di ogni piu” sana ed autentica Civiltà.

Il recupero della memoria del nostro passato, passerà anche attraverso il doveroso rispetto delle illustri personalità del passato che grande lustro hanno dato a Gioia del Colle intermini di famiglia, educazione e cultura.

\sphinxstylestrong{Giovanni Carano-Donvito, Ricciotto Canuto, Francesco Romano, Enrico Castellaneta, Renato Javarone, Giuseppe Labrocca, Raffaele Van Westerhout, Cristoforo Finto, Paolo Cassano, Cesare Soria, Francesco Paolo Losapio, Donato d’Eramo, Dorotea Taranto- Minei, Fortunato Matarrese, Franco Galli, Alfredo Pagano, Cesare Svelto, Melina Procino, Costantino Colacicco, Mario Girardi, Luigi Tosco, Lina Eramo, Donato Boscia, Annetta Serino, Vincenzo Oliva, Fratelli Capurso, Armando Celiberti, Mimmo Castellano} , sono solo alcuni del \sphinxstyleemphasis{grandi} della nostra città il cui ricordo sarà doveroso rinverdire non solo al nostro paese, ma anche ai giovani studenti;
la riscoperta delle gesta dei grandi sarà uno dei tanti obiettivi cui la Pubblica Amministrazione dovrà mirare, con il giusto rispetto nei confronti di tutto coloro i quali hanno reso alla città contributi notevoli nel campo della economia, della scienza, dell’insegnamento e dell’arte.

Il compito sarà esaltarne il ricordo indicando l’esempio e la laboriosità alle nuove generazioni per renderli consapevoli delle tracce di chi - prima di noi - ha seminato grandemente per donare al nostro Paese studio e fortuna.

Importante sarà anche scoprire ed esaltare le meravigliose architetture delle masserie e ville di campagna che numerose insistono nei dintorni dell’agro gioiese: \sphinxstylestrong{Masseria Gigante, Soria, San Pietro, Cassano, Rosati, Eramo, Ciavea, Surico, D’Onghia, Mandorlamara, Fatalone, Vallata, Colombo} andranno tutte indicate su apposite mappe illustrative all’interno delle quali verrà descritta la loro ubicazione e la loro storia.

Anche i Palazzi gentilizi delle famiglie gioiesi verranno adeguatamente esaltati, indicandone all’uopo epoche, costruttori, stili e proprietà: \sphinxstylestrong{Palazzo Surico, Cassano, Pagano, Romano, Monte, Jacobellis, Tateo, Nico, Pavove, Rizzi, Cirsella, Serino, Soria, Tateo, Carnevale, Capurso, Carano-Donvito, Tarsia-Incuria, Boscia ed Eramo}, tutti godranno di una nuova rinascita e giusta esaltazione.

Il tutto senza dimenticare i tra importanti Conventi di Gioia del Colle: \sphinxstylestrong{Sant’Antonio, San Domenico e San Francesco}, insieme alla esaltazione degli slarghi e degli archi che numerosi insistono all’interno del nostro meraviglioso centro storico.

Direttrice principe dell’Assessorato alla Cultura del Comune di Gioia del Colle, sarà comunque quella di indicare con puntualità e precisione le linee guida ispiratrici volta per volta di percorsi ed iniziative, in modo tale da tracciare solchi all’interno dei quali operare e mirare in vista del comune obiettivo di rendere la Città di Gioia del Colle eccellenza nel campo delle Arti, delle Lettere e delle Scienze.


\section{Teatro «ROSSINI»}
\label{\detokenize{cultura:teatro-rossini}}
\noindent{\hspace*{\fill}\sphinxincludegraphics[width=1.000\linewidth]{{teatro_rossini}.jpg}\hspace*{\fill}}


\subsection{Organizzazione, Gestione e Direzione}
\label{\detokenize{cultura:organizzazione-gestione-e-direzione}}
Il \sphinxstylestrong{Teatro Comunale «Rossini»} preziosa ed insostituibile risorsa per la cultura gioiese, avrà la capacità di divenire pregiato contenitore per eventi e rappresentazioni teatrali di livello nazionale ed internazionale.
Grazie alla sua imponente costruzione storica che lo rende immobile di notevole rilevanza architettonica, il Teatro Comunale Rossini sarà destinato ad essere il centro pulsante di importanti iniziative, nonchè vera e propria fucina di produzione teatrale. .

Il Teatro Comunale Rossini, con il suo nuovo assetto organizzativo e direzionale infatti, si proporrà non solo come mero acquirente di spettacoli e/o pacchetti da proporre all’interno dei vari cartelloni di stagione, quanto piuttosto in grado di patrocinare e produrre imprese culturali in proprio, da essere destinate alla promozione, vendita e commercializzazione.
Al fine di perseguire tale obiettivo, sarà opportuna la istituzione di una vera e propria Scuola di Teatro in grado di formare «addetti ai lavori» i quali, in forza di proprie precipue competenze e specializzazioni acquisite, potranno realizzare, sempre per conto del Comune di Gioia del Colle, e per esso del Teatro Comunale, lavori e proposte culturali destinate alla commercializzazione e distribuzione, anche per il tramite di realizzazioni filmiche, sempre prodotte dal predetto Ente, dando priorità e supporto alle associazioni gioiesi già operanti sul territorio.

Per la organizzazione del Teatro Comunale Rossini, sarà opportuno individuare ed esaltare figure esperte per ricoprire ruoli al proprio interno i quali, individuati in base a precipue loro competenze, collaboreranno insieme alle Associazioni Culturali e di Servizio che faranno apposita istanza per partecipare alla gestione della nostra importante e consolidata realtà teatrale che è quella del «Rossini» di Gioia del Colle; all’uopo andranno adeguatamente coinvolte ed esaltate tutte le realtà e le competenze locali che con competenza ed esperienza, ma anche con spirito di servizio e di liberalità, si sono sempre occupate di cultura e di divulgazione della stessa.

Il Teatro Comunale Rossini, per il tramite dell’Assessore alla Cultura e delle Direzioni Artistica e Tecnica, andrà a collaborare ed interagire con le Scuole presenti nel nostro territorio, all’uopo organizzando appositi corsi didattici al loro interno, ma anche con le Associazioni Culturali e di Servizio, con la Consulta per la Cultura, con le Scuole nonchè con singole personalità che - dotate di particolare propensione per l’Arte e la Cultura- andranno ad essere individuate come interlocutori necessari per future forme di collaborazione.
Anche per il campo della musica sarà opportuno avvalersi di figure esperte nel suo campo le quali, in base a loro proprie precipue competenze, potranno coadiuvare con la Direzione del Teatro e con l’assessorato alla Cultura al fine di stilare una programmazione musicale adatta alla città ed ai suoi potenziali fruitori, all’uopo spaziando nei campo della musica sinfonica, operistica, leggera, da camera, popolare e jazz, e predisporre periodicamente proposte e cartelloni.


\section{Castello ed Area Archeologica di Monte Sannace}
\label{\detokenize{cultura:castello-ed-area-archeologica-di-monte-sannace}}
\noindent{\hspace*{\fill}\sphinxincludegraphics[width=1.000\linewidth]{{castello}.jpg}\hspace*{\fill}}


\subsection{Utilizzo e Fruibilità}
\label{\detokenize{cultura:utilizzo-e-fruibilita}}
Al fine di una maggiore, piu” giusta ed opportuna fruibilità di tutti gli ambienti del Castello Normanno - Svevo di Gioia del Colle, nonchè dell’area relativa al sito Archeologico di Monte Sannace situato sulla S.P. Gioia - Turi, sarà necessario istituzionalizzare un formale accordo contrattuale tra Comune di Gioia del Colle (e per esso il Sindaco e l’Assessore alla Cultura pro - tempore) e la Direzione del Castello Normanno - Svevo; a tale accordo sarà opportuna la partecipazione di altri Enti necessari partner del protocollo: tra essi il Polo Museale, il Museo Archeologico Nazionale di Gioia del Colle, il MIBACT e la società Nova Apulia, la quale attualmente gestisce lo spazio di piano-terra destinato a bar, rivendita di libri e book shoop.

Il rapporto obbligatorio a stipularsi andrà a disciplinare con attenzione tempi, condizioni,. modalità ed eventuali costi per l’utilizzo da parte della Pubblica Amministrazione, non solo del Castello e dei suoi ambienti, ma anche del sito archeologico, offrendo la possibilità al nostro Comune di ivi organizzare eventi e proposte di altro livello culturale ed artistico.

Tale protocollo d’intesa avrà l’obiettivo, rendendo maggiormente fruibili le aree del maniero federiciano gioiese e di Monte Sannace, di attrarre un maggior flusso di turisti nella nostra Città, per il tramite di occasioni di aggregazioni di pubblico e di turisti. Confluendo in Gioia del Colle attratti da iniziative culturali di ampio respiro, e comunque in sintonia con quelle che sono le nostre tradizioni e la nostra storia, gli avventori potranno usufruire delle attività commerciali ivi esistenti, acquistando all’uopo prodotti tipici della nostra gastronomia e degustando menu” a tema presso i nostri ristoranti ed osterie, in tal modo muovendo economie e tipicità nostre squisitamente locali.

Nelle adiacenze dell’antico sito archeologico di \sphinxstylestrong{Monte Sannace}, si erge la Chiesetta dell’Annunziata, affidata alle cure della Parrocchia Santa Maria Maggiore di Gioia.
E” tradizione che, nelle due domeniche successive alla Pasqua, dopo la celebrazione liturgica svolta al mattino, si svolge la tradizionale ed antica \sphinxstyleemphasis{Passata al Monte}; in tale occasione i fedeli effettuano tre giri intorno alla Chiesa, terminando il tutto con l” «affidamento» del «passato» alla Madonna; il tutto in segno di devozione e di intercessione. Sarà cura della Amministrazione esaltare questa suggestiva tradizione popolare organizzando celebrazioni di concerto con la Parrocchia.

Anche la festa del primo Maggio a Montursi in occasione della ricorrenza di San Giuseppe lavoratore sarà rivalutata ed adeguatamente esaltata di concerto con la locale Associazione e tutti gli abitanti della storica contrada dell’agro gioiese.


\section{Consulta per la Cultura}
\label{\detokenize{cultura:consulta-per-la-cultura}}

\subsection{Finalità ed Istituzione}
\label{\detokenize{cultura:finalita-ed-istituzione}}
La Consulta Comunale della Cultura del Comune di Gioia del Colle, il cui regolamento è già in essere presso il Comune di Gioia del Colle, ma giammai reso operativo nella sua propria precipua organizzazione, deve perseguire al suo interno la promozione ed il coordinamento delle attività culturali locali in stretta collaborazione con Associazioni, Sodalizi e Scuole operanti nella nostra Comunità.
La Consulta inoltre avrà il compito di fare emergere le esigenze e i bisogni della cittadinanza e dei singoli in riferimento alla cultura; di stimolare e favorire tutte quelle iniziative in grado di potenziare le attività della cultura e dello spettacolo; di coordinare l’associazionismo culturale presente sul territorio esaminandone le problematiche e ricercando le più appropriate soluzioni alle stesse; di promuovere l’attività delle associazioni elaborando strategie comuni per la valorizzazione del patrimonio culturale, delle iniziative letterarie, scientifiche, teatrali e musicali programmate nel territorio comunale; calendarizzare gli eventi da svolgersi presso i contenitori culturali; pubblicizzare gli eventi culturali patrocinati dal Comune anche attraverso l’utilizzo di un apposito spazio all’interno del rinnovato sito istituzionale del Comune, promuovere attività di collaborazione tra le varie associazioni.

La Consulta per Cultura avrà l’obbligo di operare in stretta sinergia e collaborazione con l’Assessorato alla Cultura, il cui esponente ne farà parte di diritto, ed avrà la facoltà di partecipare a tutte le riunioni e comunque essere informato in merito alle decisioni ed iniziative intraprese.
Alla Consulta Comunale della Cultura, nel cui coordinamento deve essere coinvolta la locale «Pro Loco», quale Associazione Turistica espressamente riconosciuta dall’Ente Comunale, e per la cui attività di promozione turistica si auspica maggiore operatività, potranno divenire parti tutte le Associazioni Culturale e di Servizio presenti sul territorio che ne avranno palesato espressa adesione a seguito di formale invito da parte dell’Ufficio Cultura, avrà la stessa durata di quello amministrativo del Consiglio Comunale che l’ha istituita.


\section{I Contenitori culturali}
\label{\detokenize{cultura:i-contenitori-culturali}}

\subsection{Palazzo Serino, Palazzo Tateo, Palazzo S.Antonio, ex Distillerie}
\label{\detokenize{cultura:palazzo-serino-palazzo-tateo-palazzo-s-antonio-ex-distillerie}}
\noindent{\hspace*{\fill}\sphinxincludegraphics[width=1.000\linewidth]{{cassano}.jpg}\hspace*{\fill}}

Una integrale opera di ripensamento, riqualificazione e rimodulazione degli spazi e degli ambienti di proprietà comunale meriteranno anche gli immobili di proprietà della pubblica amministrazione, sempre nel rispetto della progettazione e delle delibere già allo stato esistenti.
Frutto di lasciti testamentari e di donazioni in favore del nostro Municipio da parte di Famiglie benemerite di Gioia del Colle, i Palazzi di proprietà del Comune di Gioia del Colle saranno oggetto di studio e di attenzione, nell’ottica non solo di un razionale utilizzo degli stessi, ma anche di una opera di ristrutturazione ed ammodernamento.

Cura della Amministrazione sarà altresi” individuare professionisti specializzati nel settore in grado di inviduare bandi regionali e/o europei atti a finanziare progetti, opere ed idee di rilevanza culturale, ma anche tesi alla ristrutturazione di immobili di proprietà pubblica, ivi elaborando progetti di utilità, crescita e progresso comune.

Ulteriore attività sarà quella tesa ad un utilizzo sociale dei grandi ambienti degli immobili posti su \sphinxstylestrong{via Paolo Cassano (ex LUM), di Palazzo S.Antonio e della ex Distilleria} su via Prov.le Gioia - Santeramo: per questi ultimi, considerata la vastità delle loro aree sarebbe opportuno, previa verifica dello stato di conservazione e manutenzione degli stessi, istituire una apposita commissione comunale finalizzata ad elaborare piani economici, finanziari e sociali in relazione al loro futuro ed immediato utilizzo in favore della Comunità.

Auspicabile sarà il mantenimento e l’ammodernamento dell’attuale \sphinxstylestrong{Info Point} il quale assicurerà la presenza di personale comunale durante le ore lavorative,ma soprattutto nel corso dei mesi estivi in cui il flusso turistico è maggiore; il tutto con personale a reperirsi tra le varie Associazioni di Gioia del Colle che ne faranno apposita richiesta, e che potrebbero gestire il punto informativo fornendo informazioni su: turismo, luoghi d’interesse storico,culturale ed architettonico, bar, ristoranti, pizzerie, masserie, chiese, luoghi, monumenti, eventi e/o iniziative culturali, luoghi d’aggregazione e palazzi padronali.


\section{Pittori locali, Musica, Arti figurative}
\label{\detokenize{cultura:pittori-locali-musica-arti-figurative}}

\subsection{Occasioni, Sistema, Promozione artistica}
\label{\detokenize{cultura:occasioni-sistema-promozione-artistica}}
La Città di Gioia del Colle ha la fortuna di annoverare al proprio interno una moltitudine di personalità particolarmente versatili nel campo delle lettere, della musica e delle arti figurative. Considerata tale ricchezza di intelligenze, sarà quantomai opportuno un coinvolgimento corale di tutti con l’obiettivo di rendere Gioia vera e propria «città della bellezza».

Perseguire l’abbellimento del centro storico attraverso la promozione di attività ed iniziative benemerite come «Le Porte dell’imperatore» ed in generale della parte piu” antica e suggestiva di Gioia, nonchè la creazione di un circuito artistico, magari in sinergia con altri Comuni limitrofi, destinato alla realizzazione ed organizzazione di appuntamenti e mostre periodiche.

Opportuno sarà anche favorire un proficuo interscambio collaborativo con le nostre attività commerciali (bar, ristoranti, pizzerie); le stesse, di concerto con i musicisti e le band locali opportunamente censite e riunite insieme in un apposito Albo, ben potrebbero organizzare manifestazioni all’aperto favorendo in tal modo la realizzazione di cartelloni estivi per mostre e concerti con artisti gioiesi, all’uopo agevolati riguardo tributi per tasse di occupazione di suolo pubblico, proprio in virtu” della loro capacità organizzativa di eventi aperti alla fruizione pubblica.
Il coinvolgimento degli artisti locali sarà utile anche al fine di una piu” adeguta riqualificazione delle periferie attraverso la loro impronta artistica che potrebbe essere lasciata come segno tangibile della loro arte, in tal modo andando a donare tracce di cultura a zone notoriamente trascurate della nostra città.

Anche la locale nascente pinacoteca comunale ad ubicarsi presso l” immobile attualmente occupato dall’INPS, potrà accogliere non solo grandi mostre, ma anche personali di pittori e scultori locali, fungendo anche da laboratori aperti al pubblico per eventuali corsi o sperimentazioni artistiche.
L’Amministrazione, infine, non farà mancare il proprio supporto e patrocinio in favore del glorioso Concerto Bandistico della Città di Gioia del Colle, che tanta fama ha diffuso nel mondo per le sue celeberrime esecuzioni musicali, nonchè per la diffusione della cultura letteraria, anche attraverso la organizzazione in proprio di rassegne letterarie in sinergia con scrittori, scuole e Case Editrici.


\chapter{Scuole e Pubblica Istruzione}
\label{\detokenize{scuola:scuole-e-pubblica-istruzione}}\label{\detokenize{scuola::doc}}
Numerosi e prestigiosi da punto di vista storico e della memoria culturale sono gli edifici scolastici del territorio di Gioia del Colle.
\sphinxstylestrong{Le scuole elementari San Filippo Neri e Mazzini, i Licei Classico Virgilio e Scientifico Canudo, l’Istituto Tecnico Galilei}, testimoniano il prestigio e la qualità sia degli studenti che nei decenni si sono avvicendati nelle Aule Scolastiche, sia degli illusti Insegnanti - veri e propri Maestri Vita, Etica e Moralità.

Il Comune di Gioia del Colle ha il dovere di tributare alla Scuola, autentica fucina di cultura, la giusta attenzione che si deve alla Istruzione Pubblica, consapevoli che il livello di istruzione della popolazione, è direttamente proporzionale al grado di civiltà ed educazione della stessa.

\sphinxstylestrong{Recuperare gli immobili scolastici tramite interventi mirati non solo alla loro manutenzione}, ma anche finalizzati ad un loro più intelligente utilizzo come \sphinxstylestrong{Sale Meeting, conferenze, eventi, happening, e quant’altro, in grado di creare appuntamenti e/o occasioni per incontri e dibattiti}; tale partecipazione potrà essere meglio realizzata per il tramite del reperimento e successiva attuazione e progettazione di bandi regionali, all’uopo appositamente reperiti ed utilizzati.
A tal fine sarà coltivata ancora maggiormente la \sphinxstylestrong{Consulta della scuola}, per il tramite della individuazione di referenti delle stesse (docenti e discenti) con il Comune di Gioia del Colle, ed in particolare con l’Assessorato alla Cultura, in modo tale da far recepire all’Ente Pubblico istanze, necessità e proposte tali da creare sinergie positive in grado di produrre progetti ed aspettative condivise.

\sphinxstylestrong{L’Amministrazione Comunale avrà il compito di assicurare l’idoneità dei contenitori scolastici presenti nella città, mettendo a disposizione le necessarie risorse economiche per le verifiche delle strutture e dei soffitti in modo tale che le famiglie sappiano i loro figli in luoghi sicuri}.
Sarà altresì importante verificare costantemente la sicurezza dei vari plessi scolastici, al fine di garantire la serenità e la incolumità dei fruitori degli stessi, \sphinxstyleemphasis{unitamente all’aggiornamento dei supporti informatici delle scuole e i collegamenti internet*} consentendo ai nostri giovani di tenere il passo con il progresso tecnologico, nonché dotare la scuola di strumentazioni adatte per qualità e quantità a consentire lo svolgimento di manifestazioni culturali, all’uopo studiando forme di accesso agevolate agli spettacoli in calendario presso il Teatro Rossini, nonchè sensibilizzando la popolazione scolastica alla partecipazione agli eventi culturali della città.


\section{Innovazione}
\label{\detokenize{scuola:innovazione}}

\subsection{Scuola Senza Zaino}
\label{\detokenize{scuola:scuola-senza-zaino}}
Gli spazi dell’aula e della scuola, in una scuola progettata per \sphinxhref{https://www.scuolasenzazaino.org}{Senza Zaino} ,sono organizzati per concretizzare l’idea di Comunità e permettere l’incontro e il lavoro condiviso dei docenti e degli allievi.
Lo spazio-aula è strutturato in aree e prevede un luogo di incontro per gli allievi, denominato agorà o forum, particolarmente significativo per la comunità-classe.




\chapter{Sport}
\label{\detokenize{sport:sport}}\label{\detokenize{sport::doc}}
\noindent{\hspace*{\fill}\sphinxincludegraphics[width=1.000\linewidth]{{sport}.jpg}\hspace*{\fill}}

\sphinxstylestrong{Lo sport è momento di educazione, socializzazione e cura.}
Lo sport è uno strumento per lo sviluppo completo e armonico della personalità dei nostri giovani, quindi del nostro futuro, le associazioni e società sportive hanno un ruolo fondamentale in questo percorso di crescita, saranno il nostro punto di riferimento per tutte le nostre politiche di diffusione della pratica sportiva.
I parchi con attrezzature e spazi sportivi, insieme alle piste ciclabili sono luoghi ideali in cui fare pratica sportiva non agonistica, soprattutto per tutti coloro che vogliono praticare sport senza essere associati.
L’importanza dell’attività motoria e sportiva non è riconducibile al solo benessere fisico, ma essa assume una fondamentale funzione di carattere culturale, sociale e formativa nonchè strumento di integrazione.

La pessima politica di gestione e la manutenzione degli impianti sportivi negli ultimi anni, ha creato una situazione di disagio e seri problemi senza precedenti per tutte le società sportive.
Alcune discipline sportive sono state addiritture costrette a trasferirsi in altre città per poter disputare le gare e allenamenti del proprio campionato agonistico.
E” necessario riportare ad un \sphinxstylestrong{livello base} di efficienza gli impianti, affinchè le società e associazioni sportive gioiesi possano ritornare a svolgere le loro attivitò nella nostra città.

Sarà necessario pianificare una attenta politica volta all’incremento della pratica sportiva attraverso:
\begin{enumerate}
\def\theenumi{\arabic{enumi}}
\def\labelenumi{\theenumi .}
\makeatletter\def\p@enumii{\p@enumi \theenumi .}\makeatother
\item {} 
Una corretta pianificazione di interventi strutturali ordinari e straordinari mediante il ricorso alle molteplici fonti di finanziamento pubbliche e private;

\item {} 
Incentivare e collaborare alla realizzazione di manifestazioni sportive costituenti un momento importante di diffusione delle discipline sportive oltre che momento di promozione turistico commerciale;

\item {} 
Razionalizzare le tariffe per l’uso delle strutture col compito precipuo di allargare soprattutto alle categorie di cittadini che versano in difficoltà economioche e sociali;

\item {} 
Utilizzare lo sport quale strumento attivo di prevenzione socio-sanitaria e di crescita relazionale e culturale;

\item {} 
Promuovere lo sport all’interno delle scuole consentendo l’utilizzo delle strutture durante le ore scolastiche;

\item {} 
Coordinare e promuovere la cooperazione delle associazioni/società sportive presenti sul territorio al fine armonizzare ed equilibrare i principali filoni dello sport giovanile, sport per tutti e sport di alto livello;

\item {} 
Garantire l’adeguata rappresentanza ed il confronto continuo alle società sportive al fine di rendere partecipi dei procedimenti decisionali gli operatori;

\item {} 
Attivare collaborazioni con il mondo imprenditoriale per lo sviluppo degli impianti sportivi, il loro utilizzo in ambito pubblicitario nonchè per la promozione di spettacoli.

\item {} 
Consentire alle società sportive il pieno utilizzo e l’accesso alle strutture sportive nelle ore prestabilite, prevedendo altresì l’individuazine di un responsabile per ciascuna di essa.

\item {} 
Garantire la fruizione giornaliera della pista di atletica del campo Martucci per svolgere liberamente le attività di Running.

\item {} 
Promuovere una collaborazione delle società sportive che svolgono la medesima attività al fine di istituire un vivaio giovanile idoneo a consentire la crescita agonistica senza lasciare il territorio, riconoscendo adeguate premialità a chi opererà in questa direzione.

\end{enumerate}

\noindent{\hspace*{\fill}\sphinxincludegraphics[width=1.000\linewidth]{{atletica}.jpg}\hspace*{\fill}}


\section{PalaKuznetzov}
\label{\detokenize{sport:palakuznetzov}}
Il Palazzetto PalaKuznetzov ha una pavimentazione omologata solo per il Volley, sarà necessario cambiarla con una superficie multifunzionale, adatta anche per altre discipline sportive.


\section{Partenariato Pubblico-Privato}
\label{\detokenize{sport:partenariato-pubblico-privato}}
Attivare un percorso per arrivare a un partenariato pubblico-privato per la riqualificazione degli impianti sportivi.


\subsection{Modello ESCo}
\label{\detokenize{sport:modello-esco}}
Valutare la sperimentazione del \sphinxhref{https://www.qualenergia.it/articoli/20140930-i-modelli-di-business-per-efficienza-energetica/}{modello di business ESCo} che permette di realizzare l’intervento %
\begin{footnote}[4]\sphinxAtStartFootnote
Es. Luci a Led nel Palazzetto per l’abbattimento dei costi di gestione
%
\end{footnote} assumendo tutti i rischi e costi dell’opera al privato e liberando il Comune da ogni onere organizzativo, finanziario e gestionale.
Condividere il risparmio con il privato che otterrà un beneficio economico per tutta la durata del contratto in misura proporzionale al risparmio conseguito.


\section{Organizzazione degli spazi}
\label{\detokenize{sport:organizzazione-degli-spazi}}
La distribuzione attuale delle ore d’utilizzo degli impianti sarà ridimensionata e calibrata considerando altri criteri più ristrettivi di quelli in vigore con l’attuale regolamento.
L’obiettivo sarà quello di stimare con più precisione l’effettivo uso di ogni impianto, affinchè si possano ottimizzare i costi di gestione.
I criteri di merito delle associazioni devono tener conto principalmente dei seguenti parametri:
\begin{itemize}
\item {} 
l’anzianità di affiliazione al \sphinxhref{https://www.coni.it}{CONI};

\item {} 
il numero di atleti tesserati praticanti e agonistici secondo il \sphinxhref{https://www.coni.it/it/registro-societa-sportive.html}{Registro delle Associazioni sportive};

\item {} 
il numero effettivo di atleti praticanti per ogni società sportiva;

\item {} 
esenzione del pagamento per ragazzi socialmente svantaggiati e progetti di integrazione sociale;

\end{itemize}


\subsection{Gestione degli impianti}
\label{\detokenize{sport:gestione-degli-impianti}}
Il contratto attuale per la pulizia e gestione degli impianti sportivi non è sufficiente a soddisfare le esigenze delle società sportive, dovrà essere revisionato considerando il numero di ore effettive di utilizzo degli impianti rispetto alla reale esigenza delle associazioni sportive


\subsection{Ufficio Sport}
\label{\detokenize{sport:ufficio-sport}}
L’ufficio Sport sarà riorganizzato in una nuovo organigramma di servizi comunali, affinchè possa gestire gli spazi con maggiore efficienza.


\section{Nuove strutture sportive}
\label{\detokenize{sport:nuove-strutture-sportive}}
Avviare processi di collaborazione e dialogo con la Regione Puglie e CONI per reperire finanziamenti destinati alla rigenerazione e all’utilizzo multidisciplinare delle strutture esistenti.

\noindent{\hspace*{\fill}\sphinxincludegraphics[width=1.000\linewidth]{{tensostruttura}.jpg}\hspace*{\fill}}

\noindent{\hspace*{\fill}\sphinxincludegraphics[width=1.000\linewidth]{{rugby}.jpg}\hspace*{\fill}}


\chapter{Agricoltura}
\label{\detokenize{agricoltura:agricoltura}}\label{\detokenize{agricoltura::doc}}
\noindent{\hspace*{\fill}\sphinxincludegraphics[width=1.000\linewidth]{{agricoltura}.jpg}\hspace*{\fill}}

\sphinxstylestrong{Il Comune deve essere di sostegno alla promozione del proprio territorio in tutte le sue espressioni dalle più tradizionali a quelle più avanzate e innovative.}

La tipicità dei prodotti locali insieme alla loro qualità, sicurezza alimentare e sostenibilità ambientale sono le variabili su cui fondare una politica di marketing territoriale che comprenda anche la valorizzazione delle tradizioni e prodotti tipici locali.

Le produzioni agroalimentari locali rappresentate da Consorzi e Associazioni di Produttori, saranno sostenute per privilegiare il contatto diretto tra produttore e operatore, sia buyer oppure HO.RE.CA attraverso eventi, sagre e incontri per favorire la ricerca nel settore con Università, Spin-Off e StartUP innovative.

Incentivare l’innovazione tecnologica applicata all’agro-zootecnia per potenziare la tracciabilità agroalimentare, semplificare la burocrazia, garantire la qualità e sicurezza dei prodotti lattiero-caseari nel rispetto della fertilità del suolo, riduzione delle emissioni di CO2 per valorizzare la filiera lattiero-casearia.


\section{AgroCittà}
\label{\detokenize{agricoltura:agrocitta}}
Gli allevatori e produttori dell’agro gioiese hanno diritto ad avere servizi efficienti e non devono più sentirsi trascurati e abbandonati.
La nostra coalizione intende lo spazio urbano come un territorio che ingloba anche le campagne gioiesi diventando una grande AgroCittà con al centro l’Agricoltura gioiese, importante per il nostro sviluppo sociale ed economico grazie alla sua capacità attrattiva e forte identità storica.

Essere un Agrocittà significa che l’agro gioiese deve godere degli stessi diritti di tutti cittadini gioiesi.
La riqualificazione delle aree agricole a sostegno dell’agricoltura locale sarà parte integrante dei percorsi partecipativi per redigere il nuovo Piano Urbanistico, in rete con i comuni limitrofi per co-progettare soluzioni per risolvere le criticità diffuse sulla viabilità e per il rispetto dell’ambiente, soprattutto per valorizzare il territorio a favore del turismo agricolo.


\subsection{Gestione rifiuti nell’agro gioiese}
\label{\detokenize{agricoltura:gestione-rifiuti-nell-agro-gioiese}}
Valutare l’ipotesi di una Eco Stazione, che permetteranno di risparmiare sui costi di raccolta e sulla tariffa TARI, aumentata oltre il 60\% per le zone rurali.

\noindent{\hspace*{\fill}\sphinxincludegraphics[width=1.000\linewidth]{{tivoli_rifiuti}.jpeg}\hspace*{\fill}}




\subsection{Infrastrutture e Viabilità}
\label{\detokenize{agricoltura:infrastrutture-e-viabilita}}
Ripristinare servizi essenziali per la comunità montana partendo dalle necessita come ad esempio :
\begin{itemize}
\item {} 
la sostituzione del macchinario di erogazione dell’acqua dal pozzo in località Montursi;

\item {} 
potenziamento della rete Internet per le famiglie di allevatori residenti nell’agro gioiese.

\end{itemize}


\section{Tavolo Verde, la nostra terra.}
\label{\detokenize{agricoltura:tavolo-verde-la-nostra-terra}}
Tavolo Verde di partecipazione e co-progettazione di proposte, strategie insieme alle Associazioni di categoria, \sphinxstylestrong{Pro Loco}, Consorzi di Produttori del Vino per la valorizzazione e tutela della nostra produzione agricola.


\subsection{Campagna Amica a KM0}
\label{\detokenize{agricoltura:campagna-amica-a-km0}}
\noindent{\hspace*{\fill}\sphinxincludegraphics[width=1.000\linewidth]{{mercato0}.jpg}\hspace*{\fill}}

Istituire una giornata periodica da dedicare alla Campagna Amica nel Mercato Comunale coperto e il tratto di Viale Regina Elena insieme alla Coldiretti, per promuovere le eccellenze della filiera agricola locale dal produttore al consumatore e a KM0.
Istituire attività di promozione dei prodotti locali durante lo svolgimento del mercato settimanale.
Concessione di aree pubbliche dell’attuale Mercato Coperto da destinare alle aziende agricole, per la promozione dei prodotti tradizionali locali a KM0.


\section{Tavolo Bianco, il nostro oro.}
\label{\detokenize{agricoltura:tavolo-bianco-il-nostro-oro}}
\noindent{\hspace*{\fill}\sphinxincludegraphics[width=1.000\linewidth]{{dop}.jpg}\hspace*{\fill}}

Tavolo Bianco di partecipazione e co-progettazione insieme al Consorzio di produttori della futura Mozzarella DOP di Gioia del Colle, Associazioni di categoria, \sphinxstylestrong{Pro Loco}, Università e Centri di Ricerca per sviluppare insieme strumenti per monitorare, valorizzare ed innovare la filiera lattiero-casearia, promuovere la tutela della qualità e valorizzazione del latte e dei suoi prodotti.


\subsection{Tracciabilità e BlockChain}
\label{\detokenize{agricoltura:tracciabilita-e-blockchain}}
\noindent{\hspace*{\fill}\sphinxincludegraphics[width=1.000\linewidth]{{foodchain}.jpg}\hspace*{\fill}}

Gli obiettivi dovranno assicurare la massima qualità della materia prima e dei semilavorati, la tracciabilità e sicurezza della filiera.
Il Comune di Gioia del Colle come Nodo Capofila della \sphinxstylestrong{Rete Antenna PON in Ricerca e Innovazione} sarà promotore insieme ai centri di ricerca, Università e Aziende innovative che vorranno investire sul territorio di progetti di tracciabilità agroalimentare per valorizzare il latte prodotto dagli allevatori dell’Agro Gioiese.

Il consumatore finale sarò in grado di conoscere il lavoro e l’impegno di ogni allevatore, rendendosi conto di cosa succede, dietro ogni litro di latte una mozzarella o una forma di formaggio.
Nonostante il recente obbligo di indicazione dell’origine sull’etichetta del latte, il Mercato agroalimentare odierno ha ancora molte difficoltà nel certificare l’origine e la qualità di un prodotto agricolo, che non aiuterà ad aumentare la fiducia dei consumatori e proteggere il made in Italy.

Il Comune insieme agli stakeholders ed enti di ricerca co-progetterà soluzioni per garantire una \sphinxstylestrong{tracciabilità sicura utilizzando nuove tecnologie come la ​BlockChain},​ per permettere al consumatore di accedere a tutte le notizie riguardanti un singolo prodotto grazie ad Internet.

Con la legge di Bilancio per il 2019 emerge chiaramente che l’innovazione del Paese deve avere un impronta digitale.
Infatti l’adesione dell’Italia alla \sphinxhref{https://ec.europa.eu/digital-single-market/en/news/european-countries-join-blockchain-Partnership}{Blockchain Partnership} , l’istituzione presso il Mise del \sphinxhref{https://www.mise.gov.it/index.php/it/blockchain}{tavolo di esperti per definire una strategia nazionale sulla blockchain}, fino alle nuove norme introdotte dal decreto-legge semplificazione %
\begin{footnote}[2]\sphinxAtStartFootnote
Le norme sulle tecnologie basate su registri distribuiti e smart contracts sono contenute nella Legge 11 febbraio 2019, n. 12 di conversione in legge del decreto-legge 14 dicembre 2018, n. 135, recante “Disposizioni urgenti in materia di sostegno e semplificazione per le imprese e per la pubblica amministrazione”
%
\end{footnote} e presentazione del \sphinxhref{https://www.mise.gov.it/index.php/it/198-notizie-stampa/2039329-di-maio-presenta-il-fondo-nazionale-innovazione}{Fondo Nazionale Innovazione}, indicano una strada spianata verso politiche di innovazione e potenziamento della filiera agroalimentare, favorendo la diffusione di \sphinxstylestrong{Agricoltura 4.0}.

L’Italia è leader assoluto nel campo delle eccellenze agroalimentari, sono un settore strategico per lo sviluppo dell’economia come si legge anche dal \sphinxhref{https://www.politicheagricole.it/flex/cm/pages/ServeBLOB.php/L/IT/IDPagina/394}{Report Attività 2018} del Ministero delle politiche agricole alimentari, forestali e del turismo (Mipaaft).

Il settore agricolo è in evoluzione e la \sphinxstylestrong{tecnologia blockchain} rafforzerà la tracciabilità proteggendo il futuro marchio DOP della mozzarella da frodi alimentari, ed invasioni di prodotto estero di scarsa qualità.


\subsection{Siero del Latte, da rifiuto a risorsa economica}
\label{\detokenize{agricoltura:siero-del-latte-da-rifiuto-a-risorsa-economica}}
L’obiettivo è promuovere la realizzazine di una filiera produttiva integrata cha vada dai produttori della materia prima insieme ad Associazioni di categoria e Consorzio dei Produttori della mozzarella DOP, fino alla trasformazione finale per le imprese dolciarie e farmaceutiche.


\section{Tavolo Rosso, le nostre tradizioni}
\label{\detokenize{agricoltura:tavolo-rosso-le-nostre-tradizioni}}
Tavolo di co-progettazione insieme alla \sphinxstylestrong{Pro Loco}, stakeholder, Consorzi di produttori agricoli, Coldiretti e Associazioni di categoria per la valorizzazione dei prodotti tipici locali enogastronomici e artigianali.


\subsection{Denominazione Comunale (DE.CO.)}
\label{\detokenize{agricoltura:denominazione-comunale-de-co}}
Applicazione del \sphinxhref{https://deco.readthedocs.io/en/latest/}{Regolamento ​DECO} (Approvato in Consiglio Comunale di Gioia del Colle con Delibera n.12 del 05/04/2017) per la valorizzazione delle produzioni locali affinché si costituisca una commissione insieme alla Pro Loco, di rappresentanti del settore produttivi locale, il turismo e commercio, con l’obiettivo di attivare percorsi volti alla valorizzazione e sfruttamento delle possibilità offerte da tutti i prodotti tipici enogastronomici e dell’artigianato locale.


\chapter{Cittadinanza digitale}
\label{\detokenize{digitale:cittadinanza-digitale}}\label{\detokenize{digitale::doc}}
\noindent{\hspace*{\fill}\sphinxincludegraphics[width=1.000\linewidth]{{digitale}.jpg}\hspace*{\fill}}

\sphinxstylestrong{Le città diventano smart quando sono in grado di ascoltare i cittadini e attivare la collaborazione e partecipazione per produrre servizi utili ai fabbisogni della comunità.}

La diffusione di una cultura digitale e i dati pubblici sono elementi essenziali e materia prima dell’economia digitale, garantiscono il successo dell’innovazione digitale nel territorio.
Il Comune deve essere aperto, responsive, abilitante e pronto ad una governance che sia in grado di far crescere community di innovatori e collaborazione tra territori.

La pubblica amministrazione è un soggetto attivo che crea valore pubblico, produce e utilizza una grande quantità di dati che se rilasciati pubblicamente a disposizione della collettività svelano tutto il loro potenziale di acceleratori di sviluppo, crescita e momenti importanti di confronto tra Pubblica Amministrazione e Community locale.

I dati pubblici devono essere di particolare interesse pubblico e ad alto impatto dal punto di vista sociale ed economico per poter attuare l’approccio dell’Open Government, Accountability, Partecipazione e collaborazione con creativi, soggetti competenti, Università e StartUP.

La digitalizzazione completa del Comune di Gioia del Colle è una missione fondamentale da portare a termine, \sphinxhref{https://docs.italia.it/teamdigitale/team-per-la-trasformazione-digitale/protocollo-intesa-corte-dei-conti-docs/it/stabile/protocollo.html}{altrimenti faremo un danno al territorio e all’erario}.


\section{Strategia Digitale per il Comune di Gioia del Colle}
\label{\detokenize{digitale:strategia-digitale-per-il-comune-di-gioia-del-colle}}
Le richieste dei cittadini saranno il carburante che alimenterà il motore dell’evoluzione digitale del Comune di Gioia del Colle, innescando un meccanismo virtuoso che consentirà di allentare la pressione sui costi restituendo ossigeno al bilancio comunale.

Occorrerà innanzitutto potenziare le competenze dei dipendenti pubblici, che dovranno essere in grado di affrontare la costante evoluzione della comunicazione con i cittadini.

Nell’era digitale il Cittadino dovrà essere soddisfatto in tutte le sue esigenze, con l’attuale tecnologia possiamo di sviluppare servizi digitali fruibili con smartphone e tablet, che semplificheranno la vita di ognuno di noi e velocizza il processo della completa digitalizzazione del proprio Ente.

La Strategia per una digitalizzazione totale del Comune avrà 3 (tre) punti fondamentali:
\begin{enumerate}
\def\theenumi{\arabic{enumi}}
\def\labelenumi{\theenumi .}
\makeatletter\def\p@enumii{\p@enumi \theenumi .}\makeatother
\item {} 
la creazione tra i dipendenti e gli amministratori di una Cultura Digitale attraverso un percorso di formazione e recruitment di nuove competenze;

\item {} 
L’innovazione dei sistemi informatici e software;

\item {} 
Creazione di leadership all’interno del Comune per guidare i processi di innovazione digitale.

\end{enumerate}


\subsection{Il punto di partenza}
\label{\detokenize{digitale:il-punto-di-partenza}}
Si dovrà partire sfruttando le opportunità esistenti grazie alle principali piattaforme nazionali per la cittadinanza digitale (SPID, PagoPA, ANPR), armonizzando tutti i servizi digitali già operativi, a cominciare dallo Sportello Anagrafe centrale per lo sviluppo digitale del Comune di Gioia del Colle, fino ai Tributi, Servizi Sociali, Sue/Suap per le attività commerciali, SIT per conoscere il territorio, le sue risorse ed il suo utilizzo, ed infine il Servizi Rifiuti accessibili tramite il portale \sphinxhref{https://egov.ba.it}{eGov}.


\subsubsection{SPID}
\label{\detokenize{digitale:spid}}
Il portale \sphinxstylestrong{eGov} è accessibile tramite \sphinxtitleref{SPID} (Sistema Pubblico per la gestione dell’identità digitale, utilizzabile da computer, tablet e smartphone.


\subsubsection{PagoPA}
\label{\detokenize{digitale:pagopa}}
\sphinxhref{https://www.agid.gov.it/it/piattaforme/pagopa}{PagoPA} è il sistema di pagamenti elettronici verso la Pubblica Amministrazione con il quale si possono pagare le tasse, prestazioni sanitarie, iscrizioni scolastiche e molto altro in maniera sicura e veloce.




\subsubsection{FOIA Pop}
\label{\detokenize{digitale:foia-pop}}
\sphinxhref{http://www.foiapop.it/ente/6f46f254-0ff0-40cd-be4f-f354ee3ccc88/scegli}{FOIAPOP} crea la richiesta di accesso civico semplice e generalizzato a partire dagli opendata, per permettere a tutti i cittadini di diventare parte attiva in grado di conoscere come vengono spesi i soldi pubblici e soprattutto cosa finanziano.


\subsection{Cosa faremo}
\label{\detokenize{digitale:cosa-faremo}}
Nel lungo cammino ci saranno interventi da attuare per completare il percorso di digitalizzazione entro tempi certi.


\subsubsection{Nuovo Design dei Servizi Pubblici e Social Network}
\label{\detokenize{digitale:nuovo-design-dei-servizi-pubblici-e-social-network}}
Negli ultimi anni l’accelerazione imposta dal digitale con i social network, chat, siti web user-friendly hanno ridefinito la comunicazione pubblica.
E’ necessario rilanciare Il rapporto tra cittadini e Comune attraverso un’attività comunicativa e un nuovo design dei servizi pubblici.
Il sito istituzionale e i suoi servizi pubblici collegati avranno un nuovo approccio comunicativo più reattivo in linea con i social network a disposizione dei cittadini in qualunque momento e su qualunque piattaforma, senza trasformare i suoi account social in estensioni dell’URP.


\subsubsection{ANPR e interoperabilità dei dati}
\label{\detokenize{digitale:anpr-e-interoperabilita-dei-dati}}
Il Comune di Gioia del Colle è subentrato a \sphinxhref{https://www.anpr.interno.it/portale/}{ANPR (Anagrafe Nazionale della Popolazione Residente)} il \sphinxhref{https://www.google.com/url?sa=t\&rct=j\&q=\&esrc=s\&source=web\&cd=10\&cad=rja\&uact=8\&ved=2ahUKEwjhybiJ2ZLgAhWuNOwKHW5jDr4QFjAJegQICRAC\&url=https\%3A\%2F\%2Fwww.anpr.interno.it\%2Fportale\%2Fdocuments\%2F20182\%2F50186\%2FComuni\%2Bsubentrati\%2B16112018.xlsx\%2Fa33214b3-e6e0-49fe-9c80-f1d5527fb61a\&usg=AOvVaw2GCpi4nxWtJus879iA7Rkl}{5 Novembre 2018} collegando i servizi dell’\sphinxhref{http://dgegovpa.it/Gioiadelcolle/albo/dati/20160859D.PDF}{Anagrafe alla banca dati nazionale}.

Con l’\sphinxstylestrong{ANPR} sarà possibile accedere ai propri dati e certificati anagrafici in qualsiasi momento evitando il tempo impiegato per le richieste e le conseguenti risposte degli operatori comunali. Il cittadino potrà richiedere certificati anagrafici in qualsiasi Comune, non solo in quello di residenza, oppure collegandosi alla rete in qualsiasi momento, con risparmio di tempo e risorse sulla gestione del personale comunale.

Una banda dati ANPR disponibile per tutte le aree dei servizi comunali diventando la prima innovazione concreta verso \sphinxstylestrong{una vera l’interoperabilità delle banche dati comunali}.


\subsubsection{CLOUD}
\label{\detokenize{digitale:cloud}}
Migrazione di tutti i servizi informatici verso architetture informatiche \sphinxhref{https://it.wikipedia.org/wiki/Cloud\_computing}{CLOUD} come da \sphinxhref{https://pianotriennale-ict.italia.it}{Piano Triennale per l’Informatica}.


\subsubsection{SPID per Professionisti}
\label{\detokenize{digitale:spid-per-professionisti}}
Il sito istituzionale con i servizi pubblici del Comune di Gioia del Colle saranno accessibili online tramite SPID, insieme allo SPID per professionisti che consentirà di associare ruoli professionali e iscrizione in albi.


\subsubsection{Open Source}
\label{\detokenize{digitale:open-source}}
Impegno del Comune ad adottare il più possibile soluzioni a codice sorgente aperto (open source) per favorire l’indipendenza dai fornitori, la sicurezza e l’accessibilità al patrimonio informativo pubblico


\section{BlockChain}
\label{\detokenize{digitale:blockchain}}
\noindent{\hspace*{\fill}\sphinxincludegraphics[width=1.000\linewidth]{{blockchain}.jpg}\hspace*{\fill}}

In attesa di una standardizzazione delle applicazioni \sphinxhref{https://it.wikipedia.org/wiki/Blockchain}{Blockchain} per la Pubblica Amministrazione a livello nazionale ed europeo, il Comune parteciperà a tutte i possibili progetti insieme all’Università ed Enti di Ricerca, per costituire gruppi di lavoro in rete con altri comuni che avranno il compito di elaborare ed eventualmente attuare obiettivi legati alla tecnologia blockchain.

Sviluppare possibili applicazioni legate alle tematiche della amministrazione trasparente, per tracciare le identità digitali, e le transazioni con soldi pubblici senza possibilità di occultamento o alterazione delle stesse, per prevenire l’uso illecito o improprio di fondi pubblici o privati.

La tecnologia BlockChain può aiutare la Pubblica Amministrazione ad implementare Smart Contracts e Sistemi digitali multifirma per monitorare con maggiore trasparenza e precisione i risultati e le performance di se stessa, migliorando il rapporto di fiducia con i cittadini e istituzioni, e l’interoperabilità dei suoi dati.


\chapter{Partecipazione, Innovazione e Ricerca}
\label{\detokenize{partecipazione:partecipazione-innovazione-e-ricerca}}\label{\detokenize{partecipazione::doc}}
\noindent{\hspace*{\fill}\sphinxincludegraphics[width=1.000\linewidth]{{partecipazione}.jpg}\hspace*{\fill}}

\sphinxstylestrong{Coinvolgere cittadini, organizzazioni e imprese nei processi decisionali e nella definizione di politiche locali affinchè possano contribuire con idee, conoscenze e abilità al bene comune e all’efficienza delle amministrazioni.}

La Partecipazione per migliorare l’applicazione dei modelli di Open Government e rafforzare il \sphinxstylestrong{Civic Engagement} migliorando la qualità delle politiche pubbliche locali e la ricerca e innovazione per lo sviluppo del territorio.

Il dialogo tra i cittadini e le istituzioni in tutti i livelli di attuazione, europeo, nazionale regionale e locale sarà rafforzato con strumenti partecipativi innovativi affinchè si possa strutturare una relazione di reciproca fiducia.

Il cittadino è una risorsa utente che produce contenuti e condivide, non è solo il destinatario finale dell’informazione o del processo.
Il concetto di cittadinanza non è più distinto tra cittadinanza analogica e cittadinanza digitale.

Il Comune di Gioia del Colle eroga moltissimi servizi essenziali alla collettività fortemente bisognosi di innovazione, dalla gestione efficiente della mobilità, ambiente, energia, cultura, fino al settore sociale e scolastico in cerca delle soluzioni più idonee per uscire dalla criticità dei problemi locali.

L’innovazione sociale è la chiave principale per soddisfare concretamente tutti questi bisogni e favorire lo sviluppo economico, creando contesti favorevoli affinchè le idee possano approdare sul mercato, collegando la politica in materia di ricerca e sviluppo tecnologico e quella industriale.

L’innovazione sociale diventerà un processo per favorire la contaminazione tra attori diversi che si trasforma in una vera e propria intelligenza collettiva, rendendo l’innovazione sociale qualcosa di tangibile che produce effetti e genera impatti positivi.

Il fabbisogno sociale sarà la scintilla che produrrà innovazione sociale, soprattutto in quelle aree dove non esistono risposte adeguate nel pubblico e nel privato, partendo dai bisogni primari: mangiare, abitare, lavorare, muoversi e partecipare.

L’adeguamento delle competenze dei dipendenti pubblici, verso una maggiore capacità di progettazione e di far arrivare i finanziamenti a quelle iniziative che portano sviluppo, di gestione dei progetti innovativi e di monitoraggio dei loro effetti saranno un elemento cruciale per fare innovazione nel Comune e nel territorio.


\section{Crowdfunding Civico}
\label{\detokenize{partecipazione:crowdfunding-civico}}
Aggiornare lo Statuto Comunale affinchè possa contemplare strumenti di partecipazione innovativi come il \sphinxhref{http://kupanda.net/it/crowdfunding-civico/}{Crowdfunding Civico}.


\section{Politiche Giovanili e Innovazione Sociale}
\label{\detokenize{partecipazione:politiche-giovanili-e-innovazione-sociale}}
\noindent{\hspace*{\fill}\sphinxincludegraphics[width=1.000\linewidth]{{giovani}.jpg}\hspace*{\fill}}

\sphinxstylestrong{Dedicare spazio per progetti che coniugano cultura e creatività, innovazione e coesione sociale, capacità di promuovere reti e occupazione giovanile, che recuperano spazi urbani abbandonati o degradati.}

Il territorio deve diventare uno spazio dinamico e accogliente per le occasioni di inclusione sociale e lavorativa di giovani provenienti da contesti socio-culturali differenti.
L’attivazione di iniziative giovanili dal basso saranno un’opportunità per il Comune di Gioia del Colle, per individuare soluzioni innovative a problemi di esclusione sociale, degrado e abbandono di aree e spazi urbani.


\chapter{Salute e Servizi per la Persona}
\label{\detokenize{salute:salute-e-servizi-per-la-persona}}\label{\detokenize{salute::doc}}
\noindent{\hspace*{\fill}\sphinxincludegraphics[width=1.000\linewidth]{{salute}.jpg}\hspace*{\fill}}

\sphinxstylestrong{Il Comune di Gioia del Colle subisce da anni la morte lenta della sanità pubblica, sempre più aggredita da tagli non viene più garantito l’accesso alle cure e le liste d’attesa si allungano.}

L’articolo 32 della Costituzione Italiana dichiara: \sphinxstylestrong{la Repubblica tutela la salute come fondamentale diritto dell’individuo e interesse della collettività e garantisce cure gratuite agli indigenti}.

Purtroppo questa garanzia costituzionale è compromessa dallo stato di disavanzo della sanità pubblica che causa disagi alla tutela della propria salute.
Una sanità pubblica sempre più insostenibile anche a causa del progressivo invecchiamento della popolazione che richiede sempre più tutele sanitarie con aumento di costi.


\section{Ambulatorio Sanitario Comunale}
\label{\detokenize{salute:ambulatorio-sanitario-comunale}}
Agevolare e sostenere l’apertura di un ambulatorio sanitario comunale attraverso l’adesione volontaria di medici e personale infermieristico in collaborazione con le associazioni locali promotrici e in collaborazione con ASL


\section{Servizi Socio-Sanitari}
\label{\detokenize{salute:servizi-socio-sanitari}}
L’amministrazine dovrà battersoi affinchè non perda i servizi socio-sanitari sul territorio e continui le attività di monitoraggio per garantire i sergizi essenziali garantiti per legge.


\section{Riduzione fiscali}
\label{\detokenize{salute:riduzione-fiscali}}

\subsection{Lotta alla Ludopatia}
\label{\detokenize{salute:lotta-alla-ludopatia}}
Le Attività commerciali che si impegnano a non installare le slot machine all’interno dei propri locali per i prossimi 3 anni.


\section{Baratto Amministrativo}
\label{\detokenize{salute:baratto-amministrativo}}
Sperimentare sul territorio il \sphinxhref{https://it.wikipedia.org/wiki/Baratto\_amministrativo}{Baratto Amministrativo} oramai sono diversi i Comuni che lo stanno adottando, tenuto conto che, oggi, la regolamentazione da adottare è resa più chiara nei limiti della sua applicazione alla stregua dei vari pareri dei Magistrati contabili delle Regioni dei Comuni di riferimento.
Sarà disciplinato da un Regolamento e programmato nel bilancio comunale.


\section{Gioia per il Sociale}
\label{\detokenize{salute:gioia-per-il-sociale}}

\subsection{Piani di Zona}
\label{\detokenize{salute:piani-di-zona}}
Bisogna verificare con particolare attenzione il rispetto delle normative vigenti in materia di lavoro nei servizi sociali e sociosanitari, assicurare una comunicazione innovativa per la partecipazione attiva di tutte le organizzazioni sociali.


\subsection{Tavoli sociali partecipati}
\label{\detokenize{salute:tavoli-sociali-partecipati}}
Avviare tavoli tematici di partecipazione con ASL e associazioni sociali e no-profit su temi come la disabilità e non autosufficienze, il contrasto alla violenza di ogni genere, alla povertà, percorsi di inclusione socio-lavorativa e la prevenzione e contrasto alle dipendenze patologiche.


\subsection{Pari Opportunità e Persone diversamente abili}
\label{\detokenize{salute:pari-opportunita-e-persone-diversamente-abili}}
E” intenzione di promuovere iniziative dirette a sostenere le pari opportunità e progetti di integrazione sociale dedicati a persone diversamente abili.


\section{Anziani}
\label{\detokenize{salute:anziani}}
\noindent{\hspace*{\fill}\sphinxincludegraphics[width=1.000\linewidth]{{anziani_1}.jpg}\hspace*{\fill}}


\subsection{Soggiorni organizzati}
\label{\detokenize{salute:soggiorni-organizzati}}
Recuperare qualsivoglia forma di finanziamento diretto ad organizzare Soggiorni per anziani in situazioni di disagio economico.


\subsection{Centro di Aggregazione Sociale}
\label{\detokenize{salute:centro-di-aggregazione-sociale}}
Centri di aggregazione sociale autogestiti nell’area cittadina al fine di agevolare la mobilità e consentire la facilitazione di una maggiore partecipazione.


\section{Minori}
\label{\detokenize{salute:minori}}
Con il sostegno delle Associazioni, le cooperative sociali e i volontari del Servizio Civile consolideremo le iniziative di doposcuole e di animazione educativa per i minori.


\subsection{Centri estivi}
\label{\detokenize{salute:centri-estivi}}
Promozione di spazi di incontro e socializzazione tra pari, con attività sportive, ricreative di aiuto nei compiti delle vacanze, che vedano la partecipazione attiva anche di bambini diversamente abili e/o a rischio di emarginazione sociale.


\subsection{Sostegno scolastico}
\label{\detokenize{salute:sostegno-scolastico}}
Collaborazione con gli istituti scolastici per fornire un sostegno concreto ai minori che hanno bisogni speciali di attenzione per quanto riguarda l’apprendimento e il comportamento.


\chapter{Ambiente}
\label{\detokenize{ambiente:ambiente}}\label{\detokenize{ambiente::doc}}
\noindent{\hspace*{\fill}\sphinxincludegraphics[width=1.000\linewidth]{{ambiente}.jpg}\hspace*{\fill}}

\sphinxstylestrong{Le aree verdi urbane sono una risorsa fondamentale per la sostenibilità e la qualità della vita in città, contribuendo a mitigare l’inquinamento e migliorare il benessere del cittadino.}

La principale sfida da affrontare in futuro per il territorio gioiese è trovare un equilibrio tra densità urbanistica e qualità della vita.
Gli spazi verdi sono aree urbane che permettono di migliorare sensibilmente il benessere e la salute del cittadino, sono luoghi di convivenza sociale condivisi che devono essere mantenuti e ampliati.

Il problema del traffico veicolare è una delle principali fonti di malattie croniche alle vie respiratorie, è necessario limitare notevolmente l’inquinamento prodotto insieme alle pericolose discariche abusive nelle campagne del territorio gioiese, spesso con rifiuti speciali e pericolosi come l’amianto, in aumento a causa del fenomeno fisiologico di abbandono rifiuti per la raccolta differenziata.


\section{Gioia del Colle città «Pet Friendly»}
\label{\detokenize{ambiente:gioia-del-colle-citta-pet-friendly}}
\noindent{\hspace*{\fill}\sphinxincludegraphics[width=1.000\linewidth]{{pet_friendly}.jpg}\hspace*{\fill}}

Il Comune di Gioia del Colle deve diventare pet friendly, spazi verdi vivibili per gli animali e per chi li ama, senza creare problemi agli altri cittadini che potrebbero essere intimoriti o infastiditi dalla presenza degli animali. Gli animali domestici sono in costante aumento nelle famiglie italiane, sarà necessario progettare nuovi spazi verdi attrezzati cercando di riqualificare zone altrimenti abbandonate al degrado. In questi spazi gli animali domestici, i cani in particolare, possono giocare tranquillamente e socializzare tra loro. L’accesso a queste aree verdi attrezzate sarà regolamentato e controllato affinché possano essere autorizzate al meglio.
Il Comune di Gioia del Colle contrasterà il fenomeno degli abbandoni degli animali in costante aumento.


\subsection{«Parco degli affetti» il primo cimitero per Animali}
\label{\detokenize{ambiente:parco-degli-affetti-il-primo-cimitero-per-animali}}
Individuazione di un terreno pubblico da affidare in gestione ad un’associazione tramite bando pubblico per dare degna sepoltura ai propri animali domestici.


\subsection{Canile Municipale}
\label{\detokenize{ambiente:canile-municipale}}
Completare la messa a norma del canile sanitario e ampliare le infrastrutture esistenti per fornire un servizio di adozione efficiente per la città.


\subsection{Bonus «Cane»}
\label{\detokenize{ambiente:bonus-cane}}
Con il «bonus cane» un gesto d’amore può trasformarsi anche in un buon investimento, un incentivo con benefici fiscali per chi decide di adottare un cane dal canile municipale.


\section{Air Quality}
\label{\detokenize{ambiente:air-quality}}
\noindent{\hspace*{\fill}\sphinxincludegraphics[width=1.000\linewidth]{{itea}.jpg}\hspace*{\fill}}

\sphinxstylestrong{Impedire insediamenti che possano esercitare attività di sperimentazione industriale con potenziali rischi per la salute.}

Promuovere progetti di di monitoraggio ambientale degli inquinanti atmosferici della zona Industriale in collaborazione con \sphinxhref{http://www.arpa.puglia.it/web/guest/qaria}{ARPA}, ISPRA e Università per fornire alla città dati puntuali sull’inquinamento provenienti da centraline di monitoraggio installate nella zona industriale tramite \sphinxhref{https://omniscope.me/internal/Pollution/TarantAir.iox/r/Report+ITA/\#Inquinanti}{servizi web di report dei dati} accessibili in formato \sphinxstylestrong{open data}.


\subsection{Formazione}
\label{\detokenize{ambiente:formazione}}
Attività di sensibilizzazione ambientale insieme all’ARPA, e progetti di formazione in collaborazione con Associazioni insieme alle scuole per la rilevazione dei dati ambientali.


\subsection{Lotta alle discariche abusive}
\label{\detokenize{ambiente:lotta-alle-discariche-abusive}}
Incremento dei controlli nelle campagne gioiesi del fenomeno delle discariche abusive con l’installazione di fototrappole.


\section{Un albero per ogni nato}
\label{\detokenize{ambiente:un-albero-per-ogni-nato}}
Implementare il verde urbano piantando un albero per ogni nuovo nato e per ogni bambino adottato, applicando l’\sphinxhref{http://www.gazzettaufficiale.it/eli/id/2013/02/01/13G00031/sg}{obbligo previsto per legge}.


\chapter{Mobilità sostenibile}
\label{\detokenize{mobilita:mobilita-sostenibile}}\label{\detokenize{mobilita::doc}}
\noindent{\hspace*{\fill}\sphinxincludegraphics[width=1.000\linewidth]{{mobilita}.jpg}\hspace*{\fill}}

\sphinxstylestrong{Il benessere del cittadino dipende dall’impatto della mobilità della città su di esso, più è sostenibile più benefici ci saranno sull’uomo e sull’ambiente.}

Se esistono percorsi pedonali, piste ciclabili insieme ad un efficiente trasporto pubblico il cittadino abbandonerà i mezzi privati, queste tre variabili insieme contribuiscono ad avere mobilità sostenibile.
Un servizio trasporti efficiente migliora il traffico e riduce l’inquinamento acustico e atmosferico, oltre ai benefici economici per le famiglie sulle spese legate all’utilizzo dell’auto. Purtroppo la cultura della mobilità sostenibile non è molto diffusa, oltre ad una parziale sfiducia nei confronti delle reti pubbliche di trasporti.
Per il benessere dei cittadini e dell’ambiente del territorio è necessario che il Comune di Gioia del Colle sia concretamente attivo nello sviluppo di una rete urbana di trasporti pubblici efficiente.

Le azioni fino ad oggi effettuate sul territorio di Gioia del Colle, nonostante i buoni propositi delle ultime amministrazioni, con incontri tematici, chiusure del traffico e con qualche finanziamento approvato, non hanno prodotto quel cambiamento culturale e sostanziale volto a ridurre la dipendenza dal traffico automobilistico individuale. Ciò è confermato dal netto incremento del Tasso di Motorizzazione presente in città dal 2002, ottenuto dal rapporto tra vetture circolanti ogni 1000 abitanti, i cui risultati sono i seguenti:


\section{Riduzione del Traffico}
\label{\detokenize{mobilita:riduzione-del-traffico}}

\subsection{Potenziamento delle aree a sosta regolamentata}
\label{\detokenize{mobilita:potenziamento-delle-aree-a-sosta-regolamentata}}
Potenziamento delle aree a sosta regolamentata, con disco orario, al fine di migliorare la rotazione degli stalli di sosta nel centro abitato e, contestualmente, favorire gli spostamenti con mezzi alternativi all’auto privata.


\section{Gioia del Colle nodo strategico}
\label{\detokenize{mobilita:gioia-del-colle-nodo-strategico}}
Gioia del Colle è in una posizione geografica strategica, crocevia di rete ferroviaria e stradale, in particolare per i tanti autobus che transitano nel tratto di Via Federico II di Svevia frequentato da tantissimi pendolari e studenti.
L’autobus è sempre stato un mezzo di trasporto per tutti, esistono servizi low cost flessibili e convenienti che soddisfano il passeggero e contribuiscono alla riduzione di CO2.
La fermata di Via Federico II di Svevia deve essere rigenerata affinché i viaggiatori possano essere accolti nella nostra città nel migliore dei modi.

Stipulare accordi con tutte le agenzie di trasporto per creare nuove fermate extraurbane per le destinazioni nei paesi limitrofi e Bari, come ad esempio \sphinxstylestrong{Via dei Peuceti presso il Quartiere Rinascita, zona Colle delle Gioie}.


\section{Muoversi liberamente}
\label{\detokenize{mobilita:muoversi-liberamente}}
Potenziamento del blocco del traffico domenicale nell’area \sphinxstyleemphasis{Via Roma/Piazza Plebiscito/Via Armando Celiberti/Via Carducci},


\subsection{Barriere Architettoniche}
\label{\detokenize{mobilita:barriere-architettoniche}}
Migliorare gli accessi ai marciapiedi da parte dei disabili, effettuando una mappatura degli interventi e poi procedendo con gli adeguamenti, qualora sia possibile garantire uno spostamento in sicurezza.


\section{Una città amica della bici}
\label{\detokenize{mobilita:una-citta-amica-della-bici}}
Dovrà esserci un impegno economico al fine di incentivare l’acquisto e l’utilizzo delle bici come mezzo quotidiano di locomozione.


\subsection{Piano Urbano della Mobilità Sostenibile}
\label{\detokenize{mobilita:piano-urbano-della-mobilita-sostenibile}}
Predisposizione ed adozione del \sphinxhref{http://www.gazzettaufficiale.it/eli/id/2017/10/05/17A06675/sg}{PUMS (Piano Urbano della Mobilità Sostenibile)}, strumento indispensabile per programmare la mobilità cittadina nei prossimi 10 anni e per ottenere finanziamenti europei sulla mobilità.


\subsection{Piste ciclabili}
\label{\detokenize{mobilita:piste-ciclabili}}
Valutazione della possibilità di installare percorsi ciclabili in sede fissa riducendo gli spazi di sosta delle vetture private o con rimodulazione del traffico viario.
Tali percorsi dovranno tener conto della domanda da parte della cittadinanza, evitando gli errori del passato.


\subsection{Installazione di rastrelliere}
\label{\detokenize{mobilita:installazione-di-rastrelliere}}
Istituzione di spazi di deposito pubblici per la bicicletta ed i mezzi di micromobilità in prossimità di stazioni e di fermate degli autobus a medio/lunga percorrenza, al fine di garantire l’intermodalità.
Ad esempio il passaggio alla raccolta differenziata ha permesso di eliminare i cassonetti dei rifiuti lasciando alcuni spazi che potrebbero essere utili per il decoro urbano a favore di di della mobilità sostenibile.
Installare rastrelliere di ultima generazione in questi spazi vuoti per favorire l’uso di bici in città recuperebbe spazi che altrimenti resterebbero vuoti.

\noindent{\hspace*{\fill}\sphinxincludegraphics[width=1.000\linewidth]{{rastr_1}.jpg}\hspace*{\fill}}

\noindent{\hspace*{\fill}\sphinxincludegraphics[width=1.000\linewidth]{{rastr_2}.jpg}\hspace*{\fill}}

\noindent{\hspace*{\fill}\sphinxincludegraphics[width=1.000\linewidth]{{rastr_3}.jpg}\hspace*{\fill}}

\noindent{\hspace*{\fill}\sphinxincludegraphics[width=1.000\linewidth]{{rastr_4}.jpg}\hspace*{\fill}}


\section{Il prossimo futuro}
\label{\detokenize{mobilita:il-prossimo-futuro}}
L’auto elettrica è una chiave importante per una strategia ambientale, l’ecotassa inserita nella legge di Bilancio non può essere una soluzione per decarbonizzare dei trasporti perchè rappresenta una misura spot.
L’Europa sta lavorando per creare le infrastrutture che rendano funzionale e sostenibile la mobilità su auto elettriche, così come le più grandi marche automobilistiche entro i prossimi due anni aumenteranno la produzione e la vendita di auto ibride e elettriche.


\subsection{Realizzazione colonnine di ricarica per veicoli elettrici}
\label{\detokenize{mobilita:realizzazione-colonnine-di-ricarica-per-veicoli-elettrici}}
\noindent{\hspace*{\fill}\sphinxincludegraphics[width=1.000\linewidth]{{electric_car}.jpg}\hspace*{\fill}}

Il Comune di Gioia del Colle sarà attivo a seguire le direttive e bandi europei per poter dotare il territorio di infrastrutture per la ricarica di auto elettriche.


\chapter{Sicurezza}
\label{\detokenize{sicurezza:sicurezza}}\label{\detokenize{sicurezza::doc}}
\sphinxstylestrong{La qualità della vita e la sicurezza urbana rappresentano un diritto dei cittadini.}

\noindent{\hspace*{\fill}\sphinxincludegraphics[width=1.000\linewidth]{{sicurezza}.jpg}\hspace*{\fill}}

La sicurezza deve essere garantita con l’azione coordinata di più livelli di governo e dal bilancio comunale.
Il Comune in accordo con la Prefettura deve attenuare quella sensazione di insicurezza dei cittadini, prevenendo e contrastando quei disagi legati a forme di degrado sociale, ambientale, urbano o comportamenti che ostacolano la convivenza civile.

Promuovere attività di sensibilizzazione diretta e rispetto delle regole da parte di tutti per educare alla legalità.


\section{Contrastare il Degrado Urbano}
\label{\detokenize{sicurezza:contrastare-il-degrado-urbano}}
\noindent{\hspace*{\fill}\sphinxincludegraphics[width=1.000\linewidth]{{telecamere}.jpg}\hspace*{\fill}}

Diffondere l’uso dell’App \sphinxstylestrong{Decoro Urbano} per il monitoraggio civico e per contrastare il degrado urbano; Installazione di videocamere di sicurezza per monitorare i punti più sensibili della città, le periferie e le zone con attività commerciali.


\section{Potenziamento dell’organico della Polizia Urbana}
\label{\detokenize{sicurezza:potenziamento-dellorganico-della-polizia-urbana}}
Nuovo piano di assunzioni a tempo indeterminato nel corpo della \sphinxhref{http://www.gazzettaufficiale.it/eli/id/2018/12/03/18G00161/sg}{Polizia Locale}.


\chapter{Turismo}
\label{\detokenize{turismo:turismo}}\label{\detokenize{turismo::doc}}
\noindent{\hspace*{\fill}\sphinxincludegraphics[width=1.000\linewidth]{{cultura}.jpg}\hspace*{\fill}}

\sphinxstylestrong{Per ottenere sviluppo dal Turismo e Cultura è necessaria ricerca, innovazione e soprattutto infrastrutture, con una giusta strategia di marketing territoriale.}

La cultura e il turismo generano sviluppo solo con operatori dal profilo idoneo con una formazione in grado di aumentare il potere di attrazione del territorio con prodotti turistici mirati e una strategia comune.

Il Comune di Gioia del Colle svilupperà un \sphinxstylestrong{Piano Strategico} per una politica di marketing territoriale partecipata ed efficace con scelte strategiche che valorizzano l’identità del territorio, da adeguare al mercato turistico in materia di economia del turismo e cultura, in sinergia compatibile con i comuni limitrofi.


\section{Occasioni di interesse e di curiosità}
\label{\detokenize{turismo:occasioni-di-interesse-e-di-curiosita}}
\noindent{\hspace*{\fill}\sphinxincludegraphics[width=1.000\linewidth]{{matera}.png}\hspace*{\fill}}

Anche la vocazione turistica del Comune di Gioia sarà opportunamente valorizzata da parte della Amministrazione Comunale, esaltando all’uopo realtà come Masserie, percorsi cicloturistici, Castello, Chiese, Monte Sannace, Montursi, Palazzi Padronali d’epoca.
Anche per quanto concerne la realtà tuttora in essere di ** \sphinxhref{https://www.matera-basilicata2019.it/it/}{Matera 2019} - Capitale Europea per la Cultura - ** la stessa è, anche al momento, al centro dell’interesse della futura Amministrazione la quale già sin d’ora è al lavoro per la istituzione di un vero e proprio presidio all’interno della Città di Matera in modo tale da offrire a Gioia del Colle una vetrina privilegiata per pubblicizzare le proprie realtà turistiche e culturali di rilievo.

Rilevanti percorsi dell” Agro Gioiese come i luoghi del Sergente Romano, Masseria Vallata Lebbrosario, Monte Sannace, Santuario Madonna della Scala - Via per Noci, Zona ospitante i monumenti della Archeologia Industriale (via per Santeramo) saranno oggetto di studio da parte di una apposita commissione di esperti al fine di stilare un progetto di viabilità naturalistica e monumentale finalizzato ad una maggiore riscoperta dei luoghi piu” belli e suggestivi del nostro agro.
Turismo significherà anche attenzione alle strutture alberghiere, i B\&B, Bar, Trattorie e Ristoranti- Pizzerie le quali tutte andranno rifornite di materiale pubblicitario e nuove guide (mappe) per meglio orientare il turista e veicolarlo nelle scelte e negli acquisti e per meglio orientarsi all’interno del nostro territorio.

Saranno promosse altresì visite organizzate presso Caseifici, Oleifici e Cantine di Gioia del Colle, al fine di esaltare ancora maggiormente le attività e le eccellenze della Eno-Gastronomia locale. La festa patronale di San Filippo del 26 Maggio, verrà poi considerata come patrimonio dell’Amministrazione Comunale la quale, di concerto con il Comitato Feste Patronali e la Chiesa Matrice di Gioia del Colle, farà in modo di pubblicizzare adeguatamente l’evento in modo tale da far confluire a Gioia del Colle un numero sempre maggiore di visitatori: una festa patronale riempita di cultura, storia e tradizione capace di donare a Gioia del Colle la notorietà che merita anche in termini di accoglienza e calore umano.
La cultura si sviluppa e si accresce anche attraverso la organizzazione di grandi manifestazioni di aggregazione popolare , quantomai opportune per offrire alla Città svago, distrazione ma anche conoscenza. A tal proposito sarà utile riprendere la nostra tradizionale «Festa della Mozzarella» la quale, adeguatamente ripensata nel suo stile e nella sua modalità organizzativa, andrà non solo a rappresentare la esaltazione e pubblicizzazione del nostro prodotto gastronomico «tipico», ma anche un momento di festa per tutta la città.


\section{Piano Strategico partecipato per lo sviluppo turistico}
\label{\detokenize{turismo:piano-strategico-partecipato-per-lo-sviluppo-turistico}}
Il Piano Partecipato per lo sviluppo turistico avrà l’obiettivo di creare e sviluppare la progettualità e l’operatività per il settore turistico del territorio.
Il Piano sarà co-progettato insieme a Stakeholders, Associazioni culturali, Pro Loco, operatori turistici e amministratori per raccogliere idee che metteranno al centro il turista, affinchè Gioia del Colle diventi una meta di vacanza competitiva e riconosciuta dal mercato.
Il territorio dell’Area di Gioia del Colle con i suoi patrimoni storici può attivare un potenziale inespresso, definendo strategie e proposte operative per un processo di crescita delle competenze e dei servizi digitali, al fine di creare sinergie tra gli attori pubblici e privati della filiera turistica, culturale e territoriale.


\subsection{In Rete con Comuni limitrofi}
\label{\detokenize{turismo:in-rete-con-comuni-limitrofi}}
Co-progettare Piani strategici insieme a comuni limitrofi maggiori attrattori turistici rispetto a Gioia del Colle, come ad esempio Castellana Grotte con i suoi oltre 300.000 visitatori per le sue famose Grotte di straordinaria bellezza, che da oltre 50 anni attira visitatori da tutto il mondo.
Ampliare l’offerta turistica in \sphinxstyleemphasis{sinergia} con comuni a forte vocazione turistica mediante la fruizione di informazioni tramite tecnologie informatiche, capaci di realizzare un percorso virtuale tra le bellezze della Regione e il Comune di Gioia del Colle, in modo tale da formulare una proposta turistica integrata capace di interessare un numero sempre crescente di viaggiatori, con conseguenti benefici effetti sull’economia di settore.


\subsection{Innovazione e Comunicazione}
\label{\detokenize{turismo:innovazione-e-comunicazione}}
Scoprire il territorio ed il suo patrimonio culturale attraverso l’innovazione tecnologica che possa permettere di interagire con communities, favorendo il processo di internazionalizzazione e migliorando la capacità di attrazione degli investimenti pubblici e privati.
Sviluppare una piattaforma unica di servizi digitali che consenta di attivare una comunicazione diretta di carattere turistico-culturale, per dar vita ad una nuova dimensione del rapporto fra Città e viaggiatori per abbattere il gap comunicativo tra capacità di accoglienza del territorio ed i suoi visitatori.
Una comunicazione puntuale circa: \sphinxstyleemphasis{ricettività, itinerari storici, artistici, culturali ed enogastronomici, gli eventi in programmazione}, per apprezzare e conoscere il meglio del nostro Territorio, soddisfando tutte le necessità del visitatore.


\subsubsection{BIG DATA}
\label{\detokenize{turismo:big-data}}
Capacità di comunicare ed innovare, utilizzando i \sphinxstylestrong{Big Data} a supporto delle decisioni strategiche nel turismo della citta’.
Analizzare la percezione che hanno i turisti che passano da Gioia del Colle per creare le condizioni di attrattivita’, investendo nel capitale umano che si traduce in economia.
I dati serviranno a scegliere quelle azioni che hanno il maggior impatto economico e finanziario senza stravolgere le abitudini e tradizioni locali, perche’ il territorio e’ il primo attore e motore del turismo.



\renewcommand{\indexname}{Indice}
\printindex
\end{document}